%% start of file `jdoe_classic.tex'.
%% Copyright 2006 Xavier Danaux.
%
% This work may be distributed and/or modified under the
% conditions of the LaTeX Project Public License version 1.3c,
% available at http://www.latex-project.org/lppl/.

\documentclass{moderncv}
\usepackage{url}
%\usepackage{bibtopic}
%\usepackage[round]{natbib}
%\usepackage{multibib}
%\newcites{sel,all}{Selected works, Papers}

\usepackage[style=authoryear]{biblatex}
\bibliography{my_papers}
\DeclareBibliographyCategory{crucial}

% moderncv styles
%\moderncvstyle{casual}       % optional argument are 'nocolor' (black & white cv) and 'roman' (for roman fonts, instead of sans serif fonts)
\moderncvstyle{classic}       % idem

% character encoding
\usepackage[utf8]{inputenc}   % replace by the encoding you are using
%
% personal data (the given example is exhaustive; just give what you want)
\firstname{} \familyname{}
%\title{Highly-creative cognitive researcher}
%\address{Via Ciro Menotti, 15 – 47042, Cesenatico (FC), Italy}  % for classic style
%%\address{Via Ciro Menotti, 15 – 47042, Cesenatico (FC), Italy} % for casual style
\address{Max Talanov}
\phone{+7 962 571 8296} 
\email{max.talanov@gmail.com}
%\extrainfo{\url{https://www.scopus.com/authid/detail.uri?authorId=41762833600}}
%\extrainfo{\url{https://www.researchgate.net/profile/Max_Talanov}}
\extrainfo{\url{https://scholar.google.com/citations?hl=en&user=SoUgPioAAAAJ}}
%\extrainfo{\url{www.linkedin.com/in/max-talanov-a004aa16}}
%\photo[100pt]{Talanov_Max_2012} % also optional, and the optional argument is the height the picture must be resized to

%\quote{Any intelligent fool can make things bigger, more complex, and more violent. It takes a touch of genius -- and a lot of courage -- to move in the opposite direction.}% also optional
%\quote{Video meliora proboque, deteriora sequor (Ovidio, Metamorphosis, VII, 20)}
%\renewcommand{\listsymbol}{{\fontencoding{U}\fontfamily{ding}\selectfont\tiny\symbol{'102}}} % define another symbol to be used in front of the list items

% the ConTeXt symbol
%\def\ConTeXt{%
 % C%
  %\kern-.0333emo%
  %\kern-.0333emn%
  %\kern-.0667em\TeX%
  %\kern-.0333emt}

% slanted small caps (only with roman family; the sans serif font doesn't exists :-()
%\usepackage{slantsc}
%\DeclareFontFamily{T1}{myfont}{}
%\DeclareFontShape{T1}{myfont}{m}{scsl}{ <-> cork-lmssqbo8}{}
%\usefont{T1}{myfont}{m}{scsl}Testing the font

% command and color used in this document, independently from moderncv
%\definecolor{see}{rgb}{0.5,0.5,0.5}% for web links
%\newcommand{\up}[1]{\ensuremath{^\textrm{\scriptsize#1}}}% for text subscripts

%----------------------------------------------------------------------------------
%            content
%----------------------------------------------------------------------------------
\begin{document}
%\bibliographystyle{alpha}
%\maketitle
%\makequote
\makecvtitle

%\section{Personal Information}
%\cvitem{Name}{\small Max Talanov}
%\cvitem{Passport}{\small Russia}
%\cvitem{Date of birth}{\small 12 April 1974}
%\cvitem{Contacts}{\small mtalanov@it.kfu.ru (preferred), max.talanov@gmail.com}{}
%\cvitem{Research Gate}{https://www.researchgate.net/profile/Max\char`_Talanov}
 
\section{Formal educational training}

\cvitem{2014}{\small Teaching Excellence, Carnegie Mellon University}
\cvitem{2009}{\small SEP Software Architecture}
\cvitem{2008}{\small SEP Software Requirements Analysis}
\cvitem{2007}{\small SEP Managing Software Project Team}
\cvitem{Conference and Workshop}{\small 2010 -- CEE-SECR, 2013 -- AINL, 2015 -- AOC@AMSTA, 2015 -- AINA, 2015 -- BICA, 2016 -- Fierces on BICA, 2016 -- AOC@AMSTA, 2016-- AGI, 2016 -- BICA, 2017 -- BICA, 2017 -- ICAROB, 2017 -- ICINCO. I took part in organisation of special sessions of AOC@AMSTA-2016 and AOC@AMSTA-2015. Took part in organisation of local series of Software engineering, seminars AKSES-2014.}

\section{Administrative tasks relating to education}

\cvitem{}{\small Currently I have the position of head of the neuromorphic computing and neurosimulations laboratory where with intensive involvement of students we develop multidisciplinary breaking through projects.
  Till the November of 2017 I was the deputy director for science of the Information Technology and Intelligent Systems institute (ITIS) of the Kazan Federal University, where I managed the research, grants and publication policies. For two years I was heading two chairs in ITIS the Software engineering chair and Intellectual robotics chair during this period I have established the close collaboration with institutions abroad, industrial companies to provide relevant and modern hardware and software expertise for students on highly demanded specializations. I take part in the academic committee of the KFU as well as ITIS institute.}

\section{Experience of study programmes, supervision and examinations}

\cvitem{}{\small While managing the laboratory I have built the research based department where students are involved in multidisciplinary research projects from the first day. This ``train while you do'' in-project training provides highly effective opportunity for students to start their scientific carrier. Majority of my students have their papers published in Scopus/Web of Science one of them has \href{https://www.scopus.com/authid/detail.uri?authorId=56150559300}{\emph{h-index of 2.}} I have supervised thesis works of the students of my lab and the most significant result was the papers published with the involvement of the students during their thesis works presented in international conferences and journals. I have created and done the lecturing as well as supervision of the project for the following courses:}

\cvitem{2017-Now, Kazan Federal University}{\small \emph{Introduction in Artificial intelligence}: the introduction in artificial intelligence including: machine learning and reinforcement learning, decision making, reasoning, knowledge bases and representations, natural language processing, intelligent agents. During the course students develop the AI project using principles and technologies discussed in the course.}

\cvitem{2016-Now, Kazan Federal University; 2014-2015, Innopolis University}{\small \emph{Affective computation}: the extended view on the problem of emotions reimplementation in a computational system, including: philosophical, psychological, neurobiological and computational perspectives. The course starts from birds eye view on the problem, carries on to neurobiological details of a neuromodulation and psychological models of emotions, then progresses into philosophical questions of consciousness and thinking and ends up with cognitive architectures and spiking neural networks review. \href{https://github.com/max-talanov/1/blob/master/affective_computing_course/plan.md}{\emph{Course syllabus online}}.}

\cvitem{2014-2016, Kazan Federal University}{\small \emph{Software Design Architecture}: the course for bachelor students, intended to be starting point from basics of software design and UML to principles and design patterns, with extended use of practical examples. During the course students should develop project using principles of design studied. \href{https://github.com/max-talanov/1/blob/master/software_design_course/plan.md}{\emph{Course syllabus online}}.}

\section{Methods, materials and tools}

\cvitem{}{\small I'm the head of the neuromorphic computing and neurosimulations laboratory and I use ``train while you do'' approach for the education heavily. To enroll to the laboratory one has to pass the examination test of two parts: fundamentals of neuroscience and programming basics in Python/C++. The test is used for two purposes: filter the best, identify the knowledge level of a student. If a student is accepted in the laboratory, she/he could select the research project of her/his preferences from the list of the projects of the laboratory (currently 2). We have mandatory weekly updates where we do retrospective and discuss problems of students and assign new tasks for a next week. In fact we discuss problems and progress of a student daily. During updates we also discuss publishing options as well as conferences to attend in case the team of the project achieves a significant result. If a student takes significant part in the scientific results of a project we usually provide the option to present result on the international conference. A \emph{thesis work} is special activity that requires focused work of a student in the laboratory for a half a year. During this period I insist on heavy participation in the laboratory projects every day with attentive monitoring of the progress of the thesis projects. We usually plan the phases of thesis projects as well as process creation of a thesis itself and presentation. We usually do the rehearsal of thesis presentation 3-5 times to guarantee the result of thesis presentation. For example in 2017 4 from 5 students that done the thesis works in my laboratory were assessed with highest marks in 2018 2 of 2. The software Design course marking and tracking is done via Moodle system that is closely connected to the course syllabus, that is available online. The AI course uses in-project based marking system taking in account personal impact of a student.}

\section{Educational development and applied research into teaching at university, including educational awards}

\cvitem{}{\small From the ``train while you do'' perspective I have implemented special highly competitive educational scientific laboratory that does effectively train new researchers during their bachelor/master education in the university. This way we have received the funding: according to the Russian Government Program of Competitive Growth of Kazan Federal University, by the subsidy allocated to Kazan Federal University for the state assignment in the sphere of scientific activities.}

\subsection{Grants}

\emph{Memristive spinal cord segment prosthesis} project received funding from the Russian Basic Research Fund.

\emph{Robot Dream} project received funding according to the Russian Government Program of Competitive Growth of Kazan Federal University.

\emph{NeuCogAr} project received funding from subsidy allocated to Kazan Federal University for the state assignment in the sphere of scientific activities.

\emph{Erasmus+} with Birmingham City University we have received in 2016 and 2018 for the lecturers exchange. 


\section{Reflections on your own teaching practice and future development including student evaluations}

\cvitem{}{\small I consider the students feedback as crucial part for the syllabus and contents of a course. That's why I constantly monitor the understanding processes using retrospectives of every lecture. During them students recall the contents of a previous lecture and I could monitor their understanding. As for the training in the laboratory I could identify the key features of evaluations: examination test as inbound criteria as well as weekly project monitoring are crucial for the successive research laboratory with highly percentage of students. Progressing in the direction of cutting edge research and orientation to a result involves more and more motivated students in the laboratory that supports the high quality scientific research.}

\end{document}

