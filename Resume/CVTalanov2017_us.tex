%% start of file `jdoe_classic.tex'.
%% Copyright 2006 Xavier Danaux.
%
% This work may be distributed and/or modified under the
% conditions of the LaTeX Project Public License version 1.3c,
% available at http://www.latex-project.org/lppl/.

\documentclass{moderncv}
\usepackage{url}
%\usepackage{bibtopic}
%\usepackage[round]{natbib}
%\usepackage{multibib}
%\newcites{sel,all}{Selected works, Papers}

\usepackage[style=authoryear]{biblatex}
\bibliography{my_papers}
\DeclareBibliographyCategory{crucial}

% moderncv styles
%\moderncvstyle{casual}       % optional argument are 'nocolor' (black & white cv) and 'roman' (for roman fonts, instead of sans serif fonts)
\moderncvstyle{classic}       % idem

% character encoding
\usepackage[utf8]{inputenc}   % replace by the encoding you are using
%
% personal data (the given example is exhaustive; just give what you want)
\firstname{} \familyname{}
%\title{Highly-creative cognitive researcher}
%\address{Via Ciro Menotti, 15 – 47042, Cesenatico (FC), Italy}  % for classic style
%%\address{Via Ciro Menotti, 15 – 47042, Cesenatico (FC), Italy} % for casual style
\address{Max Talanov}
\phone{+7 962 571 8296} 
\email{max.talanov@gmail.com}
%\extrainfo{\url{https://www.scopus.com/authid/detail.uri?authorId=41762833600}}
%\extrainfo{\url{https://www.researchgate.net/profile/Max_Talanov}}
\extrainfo{\url{https://scholar.google.com/citations?hl=en&user=SoUgPioAAAAJ}}
%\extrainfo{\url{www.linkedin.com/in/max-talanov-a004aa16}}
%\photo[100pt]{Talanov_Max_2012} % also optional, and the optional argument is the height the picture must be resized to

%\quote{Any intelligent fool can make things bigger, more complex, and more violent. It takes a touch of genius -- and a lot of courage -- to move in the opposite direction.}% also optional
%\quote{Video meliora proboque, deteriora sequor (Ovidio, Metamorphosis, VII, 20)}
%\renewcommand{\listsymbol}{{\fontencoding{U}\fontfamily{ding}\selectfont\tiny\symbol{'102}}} % define another symbol to be used in front of the list items

% the ConTeXt symbol
%\def\ConTeXt{%
 % C%
  %\kern-.0333emo%
  %\kern-.0333emn%
  %\kern-.0667em\TeX%
  %\kern-.0333emt}

% slanted small caps (only with roman family; the sans serif font doesn't exists :-()
%\usepackage{slantsc}
%\DeclareFontFamily{T1}{myfont}{}
%\DeclareFontShape{T1}{myfont}{m}{scsl}{ <-> cork-lmssqbo8}{}
%\usefont{T1}{myfont}{m}{scsl}Testing the font

% command and color used in this document, independently from moderncv
%\definecolor{see}{rgb}{0.5,0.5,0.5}% for web links
%\newcommand{\up}[1]{\ensuremath{^\textrm{\scriptsize#1}}}% for text subscripts

%----------------------------------------------------------------------------------
%            content
%----------------------------------------------------------------------------------
\begin{document}
%\bibliographystyle{alpha}
%\maketitle
%\makequote
\makecvtitle

%\section{Personal Information}
%\cvitem{Name}{\small Max Talanov}
%\cvitem{Passport}{\small Russia}
%\cvitem{Date of birth}{\small 12 April 1974}
%\cvitem{Contacts}{\small mtalanov@it.kfu.ru (preferred), max.talanov@gmail.com}{}
%\cvitem{Research Gate}{https://www.researchgate.net/profile/Max\char`_Talanov}
 
\section{Personal Profile}

\cventry
    {Bio}{Highly-creative cognitive researcher with broad research experience}{}{}{}{Self motivated highly productive ideas generator with broad experience in: affective computing, neurobiological simulations, bio-inspired cognitive architectures, neuromorphic computing, artificial intelligence, natural language processing, probabilistic reasoning. I'm heading projects at different laboratories for 3 years managing breaking-through projects (see below).\\
      Currently I have the position of the deputy director for science of the Information Technology and Information Systems institute (ITIS) of the Kazan Federal University, where I manage research, grants and publication policies.\\
      As well as I'm the head of neurotechnology projects at neuroscience laboratory managing multidisciplinary projects in: affective computing, computational neuroscience, neuromorphic computing (electronics, memristive devices), biologically inspired cognitive architectures.\\
      Also I have industrial experience as R\&D project leader and software design architect for 8 years in international industrial projects in Fujitsu.}

\cvitem
    {Strengths}{\small Highly creative, ideas generator, excellent at managing a research team in multidisciplinary projects, have multidisciplinary mindset in: computer science, cognitive science, neuroscience, electronics, IT, philosophy.}
%{Weaknesses}{Unable to work for a long time on a trivial tasks, has difficulties to work with demotivated team}

\section{Projects}

\cvitem{NeuCogAr (2014-now)}{\small Neurobiologically inspired cognitive architecture for simulation of neurobiologically plausible emotions (based on works of Hugo L\"{o}vheim) in a computational and robotic systems based on neural simulations. \emph{Breakthrough}: \emph{Affective computing} - first time the bio-plausible implementation of psycho-emotional states mapped to computational processes will be demonstrated. \emph{Cognitive architectures and robotics}: first time the bio-plausible emotional drives will be implemented to form behavioral strategies of an artificial system. We have already demonstrated: ``fear-like'' and ``disgust-like'' states. \emph{In collaboration with:} Jordi Vallverd\'u, Universitat Aut\'onoma de Barcelona. \href{https://github.com/research-team/neucogar}{\emph{Project site}}.}

\cvitem{Memristive brain (2016-now)}{\small The robotic part of the Robot Dream project dedicated to bio-inspired memristive implementation of a mammalian brain circuits and brain areas capable of real-time emotional processing in a robotic system, based on polyaniline memristive neurons implementation capable of inhibition and neuromodulation. \emph{Breakthrough}: \emph{Computer science, AI and robotics}: new type of hardware with new options of self-learning and adaptation real-time is implemented using bio-inspired architectures with new level of understanding of the functions of neurobiological mechanisms. \emph{Electronics}: the development of memristive direction could lead to a revolution in IT industry, triggering development of highly effective self-learning devices. \emph{Neuroscience}: the brain-computer interface has new boost with the use of memristors as the interface between living cells and not-living electronic memristors in a hybrid system. \emph{In collaboration with:} Victor Erokhin Universit\'{a} degli studi di Parma. \href{https://github.com/research-team/memristive-brain}{\emph{Project site}}.}

\cvitem{Robot Dream (2015-2017)}{\small The integration of a HPC mammalian brain simulation with a real-time robotic system. Two phase architecture based on working metaphor of a mammalian dream. The ``dream phase'' consists of emotional experience collection, processing and behavioral strategy update is implemented as neural simulations on HPC cluster. The ``wake'' robotic system is based on the memristive implementation of a mammalian brain circuits, implemented in the memristive brain project. \emph{Breakthrough}: \emph{Robotics} - first time the bio-plausible emotional driven cognitive architecture integrated with robotics embodiment will be demonstrated including sensory input and motor output neural systems. \href{https://github.com/research-team/robot-dream}{\emph{Project site}}.}

\cvitem{QME (2015-2017)}{\small Quantitative model of emotions based on neuromodulatory model of affects or ``cube of emotions'' by Hugo L\"{o}vheim implemented in the spiking neuronal network NEST. \emph{Breakthrough}: first time the 3D model of emotions will be validated on mammals that could create the basement for quantitative approaches in psychology, marketing and economics. \emph{In collaboration with:} Igor Lavrov, Mayo Clinic}

\cvitem{BioDynaMo (2015-2017)}{\small HPC framework of the bio-plausible dynamic, growing neural tissues simulations including a mammalian growing brain. \emph{In collaboration with:} CERN, New castle university, Intel, Innopolis University.}

\cvitem{GDP (2015-2017)}{The bio-plausible simulation of a hippocampus Giant Depolarizing Potentials with bio-plausible mathematical models of neurobiological processes. \emph{In collaboration with:} Roustem Khazipov, INSERM U901}

\cvitem{TU (2012-2016)}{\small Thinking-Understanding. The cognitive architecture implementing the approach of the intelligent system for a help desk automation based on: machine cognition: natural language processing (NLP), probabilistic reasoning and ``Model of six'' the model of human mental processes by Marvin Minsky}

%\cvitem{Cell radio (2015-now)}{\small Nanometers neural activity sensors to be placed, powered and radiate radio frequency in neuronal cells}
\cvitem{Menta (2011-2012)}{\small The framework for automatic software application development via genetic algorithms and NLP}

\cvitem{IDP (2010-2011)}{\small Intellectual document processing, the project for data-mining of unstructured documents via NLP and ML}

\section{Teaching Experience}

\cvitem{2017-Now, Kazan Federal University}{\small \emph{Artificial intelligence}: the introduction in artificial intelligence including: machine learning and reinforcement learning, decision making, reasoning, knowledge bases and data representations, natural language processing, intelligent agents. During the course students should develop the AI project using principles and technologies discussed in the course.}

\cvitem{2016-Now, Kazan Federal University; 2014-2015, Innopolis University}{\small \emph{Affective computation}: the extended view on the emotions and reimplementation in a computational system problem  including: philosophical, psychological, neurobiological and computational perspectives. The course starts from birds eye view on the problems, carries on to neurobiological details of a neuromodulation and psychological models of emotions, then progresses into philosophical questions of consciousness and thinking and ends up with cognitive architectures and spiking neural networks review. \href{https://github.com/max-talanov/1/blob/master/affective_computing_course/plan.md}{\emph{Course syllabus online}}.}

\cvitem{2014-2016, Kazan Federal University}{\small \emph{Software Design Architecture}: the course for bachelor students, intended to be starting point from basics of software design and UML to principles and design patterns, with extended use of practical examples. During the course students should develop project using principles of design studied. \href{https://github.com/max-talanov/1/blob/master/software_design_course/plan.md}{\emph{Course syllabus online}}.}

\section{Professional Training}

\cvitem{2014}{\small Teaching Excellence, Carnegie Mellon University}
\cvitem{2009}{\small Software Architecture}
\cvitem{2008}{\small Software Requirements Analysis}
\cvitem{2007}{\small Managing Software Project Team}

\section{Computing Professional Experience}

\cventry{current}{Deputy director for science of the Information Technology and Information Systems institute (ITIS) at the Kazan Federal University}{}{Kazan, Russian Federation}{}{Management of research activities: planning, funding, publications strategy, publicity and international networking management.}

\cventry{current}{Head of computational neurotechnology projects of the neuroscience laboratory}{}{Kazan, Russian Federation}{}{Management of the scientific group; leading multidisciplinary breakthrough research in: affective computing, neurobiologically inspired systems, computational neurobiology and management of grant applications, publication activities; managing bachelor and master students, international communications and collaborations. In collaboration with research centers: CERN OpenLab V, Lanzhou University School of information science and engineering, Samsung Research Center, Universit\'{a} degli studi di Parma, Universitat Aut\'onoma de Barcelona; scientists: Victor Erokhin, Roustem Khazipov, Jordi Vallverd\"{u}, Hu Bin, Philip Moore, Salvatore Distefano, Pei Wang.}

\cventry{2014-2015}{Lecturer at Innopolis University}{}{Kazan, Russian Federation}{}{Lecturing Affective computation course from three perspectives: Philosophical (``Model of six'' by Marvin Minsky), Psychological (Wheel of emotions by Plutchik), Neurobiological: (``Cube of emotions'' by Hugo L\"{o}vheim).}

\cventry{2006-2014}{Software Design Architect at Fujitsu GDC Russia}{}{}{}{Leading several research projects in: affective computing, machine cognition, code generation automation, natural language processing. Software Design architect in different projects based on: Scala, OpenCog.RelEx, Neo4j, OpenCog.PLN, Stanford Parser, open NARS, MinorThird, Java, EJB, Hibernate, Spring, IBM MQ, Oracle BPEL}{}{}

\section{Prizes and Awards}

2005 - Received honorary diploma of the Sun microsystems, Java projects competition, for the MILK - domain specific language for web sites creation.

\section{Grants}

\emph{Robot Dream} project received funding according to the Russian Government Program of Competitive Growth of Kazan Federal University.

\emph{NeuCogAr} project received funding from subsidy allocated to Kazan Federal University for the state assignment in the sphere of scientific activities.

\emph{Erasmus+} with Birmingham City University we have received in 2016 for the lecturers exchange. 

\section{Granted patent(s)}

Thinking-Understanding the registration of intellectual property of software product or technology.

%Selected papers
%\addtocategory{crucial}{talanov_2015,vallverdu_2016a,jordi2016importance}
%\printbibliography[title={Selected papers}, category = crucial]
 
% all papers (my_papers.bib)
\nocite{*}
\printbibliography[title={Papers}]

\cvitem{}{}

\section{Conference and Workshop Activities}

2010 -- CEE-SECR, 2013 -- AINL, 2015 -- AOC@AMSTA, 2015 -- AINA, 2015 -- BICA, 2016 -- Fierces on BICA, 2016 -- AOC@AMSTA, 2016-- AGI, 2016 -- BICA, 2017 -- BICA, 2017 -- ICAROB, 2017 -- ICINCO.

I took part in organisation of special sessions of AOC@AMSTA-2016 and AOC@AMSTA-2015. Took part in organisation of local series of Software engineering, seminars AKSES-2014. I was a general chair of HCC-2017 conference.

\section{Science-pop activities}

I took part in \href{https://www.youtube.com/watch?v=BLvS7h3kRbo}{\emph{TEDx}}, \href{https://vk.com/video-87488544_171504962}{\emph{Science Slam}} and \href{https://www.youtube.com/watch?v=sLLKxvUEA7E}{\emph{JavaDay}} science-pop events in Kazan, as well as I took part in science popular Russian resource \href{https://postnauka.ru/author/talanov}{\emph{postnauka.ru}} especially interesting dedicated to \href{https://postnauka.ru/faq/58727}{\emph{Marvin Minsky and his role in AI}}, also forbes.ru published an interview with me \href{http://www.forbes.ru/mneniya-column/288097-kak-sozdat-emotsionalnyi-iskusstvennyi-intellekt}{\emph{available here}}.

\section{Computing Knowledge}

\cvitem{Neurosimulator}{NEST, Neuron}
\cvitem{Reasoner}{OpenNARS, OpenCog.PLN, Pellet}
\cvitem{Electrophy}{OpenElectrophy, Spike viewer, StimFit} 
\cvitem{ML, NLP}{Rapid miner, Weka, OpenCog.RelEx, Stanford parser, Gate, MinorThird}
\cvitem{OWL and mind map}{Protege, FreeMind, XMind}
\cvitem{Programming}{Scala, Python, Java, Prolog, Refal, C++, Haskell, PLSQL, TSQL, Java SE, EJB, Hibernate, Spring, Seam, IBM MQ, Oracle BPEL, JBoss, Tomcat, Web Services}
\cvitem{Databases}{MSSQL, Oracle DB, PostgreSQL, Neo4j, HypergraphDB}
\cvitem{Development}{Jetbrains PyCharm, Jetbrains Idea, Eclipse, Visual Studio, Maven, Emacs, ArgoUML, Visual Paradigm, Rational Rose, Enterprise Architect, QTCreator, Oracle JDeveloper} 

\cvitem{OS}{Microsoft Windows, Linux, Mac OS}

\section{Education Background}

\cventry{2000}{PHD degree in Math modelling of Electromagnetic fields in plasma}{Kazan State Technological University, Russia}{Faculty of enterprise processes management}{}{Thesis: \emph{Electromagnetic picture of high frequency plasma}}{}

\section{Language Competence}

\cvitem{English}{Full professional proficiency}
\cvitem{Russian}{Mother tongue}

\end{document}

