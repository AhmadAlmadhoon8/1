% !TEX encoding = UTF-8
% !TEX program = pdflatex
% !TEX spellcheck = en_GB


\documentclass[english,a4paper]{europasscv}
\usepackage[english]{babel}
\usepackage{url}

\ecvname{Max Talanov}
%\ecvaddress{12 Strawberry Hill, Dublin 8 Éire/Ireland}
\ecvmobile{+7 962 571 8296}
%\ecvtelephone{+353 127 6689}
%\ecvworkphone{+353 999 888 777}
\ecvemail{max.talanov@gmail.com}
\ecvhomepage{scholar.google.com/citations?hl=en\&user=SoUgPioAAAAJ}
\ecvgithubpage{www.github.com/max-talanov}
\ecvlinkedinpage{www.linkedin.com/in/max-talanov-a004aa16/}
\ecvim{Skype}{cosmicdustman}

\ecvdateofbirth{12 April 1974}
\ecvnationality{Russian}
\ecvgender{Male}

\ecvpicture[width=3.8cm]{Talanov_Max_2012}

\date{}

\begin{document}
  \begin{europasscv}

  \ecvpersonalinfo

  \ecvbigitem{Job applied for}{Leading researcher}

  %% Projects
  
  \ecvsection{Research projects}

  \ecvtitle{2014 -- Present}{NeuCogAr}
  \ecvitem{}{Neuro-simulation of bio-plausible emotions (based on works of Hugo L\"{o}vheim) in a computational and robotic systems.}
  \ecvitem{\ecvhighlight{Breakthrough}}{\emph{Affective computing} - first time the bio-plausible implementation of psycho-emotional states mapped to computational processes is indicated. We have already demonstrated: ``fear-like'', ``disgust-like'', ``joy-like'' states.}
  \ecvitem{\ecvhighlight{In collaboration with:}} {Jordi Vallverd\'u from Universitat Aut\'onoma de Barcelona.}
  \ecvitem{\ecvhighlight{Project site}}{\href{https://github.com/research-team/neucogar}{https://github.com/research-team/neucogar}}
  
  \ecvtitle{2016 -- Present}{Memristive spinal cord segment prosthesis}
  \ecvitem{}{The bio-mimetic memristive implementation of a mammalian spinal cord circuits in a electronic system as a prosthesis, based on polyaniline memristive neurons implementation capable of inhibition and neuromodulation.}
  \ecvitem{\ecvhighlight{Breakthrough}}{\emph{Computer science}: new type of hardware with new options of self-learning and adaptation real-time is implemented using bio-inspired architectures. \emph{Electronics}: the development of memristive direction could lead to a revolution in IT industry, triggering development of highly effective self-learning devices. \emph{Neuro-rehabilitation}: the adaptive walking pattern generator implemented as electronic schematic is a first step in bio-mimetic neuronal prosthesis domain.}
  \ecvitem{\ecvhighlight{In collaboration with:}} {Victor Erokhin from Universit\'{a} degli studi di Parma, Igor Lavrov from Mayo Clinic.}
  \ecvitem{\ecvhighlight{Project site}}{\href{https://github.com/research-team/memristive-spinal-cord}{https://github.com/research-team/memristive-spinal-cord}}

  \ecvtitle{2017 -- Present}{Memristive reaction diffusion processor}
  \ecvitem{}{The bio-inspired by a dopamine neuromodulatory system the new generation of memristive processors.}
  \ecvitem{\ecvhighlight{Breakthrough}}{\emph{Computer science}: new type of hardware operating in real-time using three timescales of BZ reaction, memristive STDP learning and electronic signal transmission.}
  \ecvitem{\ecvhighlight{In collaboration with:}} {Victor Erokhin from Universit\'{a} degli studi di Parma, Andrew Adamatzky from University of West England.}
  \ecvitem{\ecvhighlight{Project site}}{\href{https://github.com/research-team/cellCircuit}{https://github.com/research-team/cellCircuit}}

  \ecvtitle{2017 -- Present}{Multi-compartment nociceptive ATP signaling}
  \ecvitem{}{The neuro-simulation of ATP signaling and spikes auto-generation in sensory fibers.}
  \ecvitem{\ecvhighlight{Breakthrough}}{\emph{Neuroscience}: First time the prolonged auto-generation of spikes triggered by ATP will be demonstrated via bio-plausible neuro-simulation.}
  \ecvitem{\ecvhighlight{In collaboration with:}} {Rashid Ginniatullin from University of Eastern Finland.}
  \ecvitem{\ecvhighlight{Project site}}{\href{https://github.com/research-team/robot-dream}{https://github.com/research-team/robot-dream}}
  % Work experience
  
  \ecvsection{Work experience}
  
  \ecvtitle{November 2016 -- Present}{Deputy director for science of the Information Technology and Information Systems institute (ITIS) at the Kazan Federal University}
  \ecvitem{}{Management of research activities: planning, funding, publications strategy, publicity and international networking management.}
   
  \ecvtitle{August 2016 -- Present}{Head of computational neurotechnology projects of the neuroscience laboratory}
  \ecvitem{}{Management of the scientific group; leading multidisciplinary breakthrough research in: affective computing, neurobiologically inspired systems, computational neurobiology and management of grant applications, publication activities; managing bachelor and master students, international communications and collaborations. In collaboration with research centers: CERN OpenLab V, Lanzhou University School of information science and engineering, Samsung Research Center, Universit\'{a} degli studi di Parma, Universitat Aut\'onoma de Barcelona; scientists: Victor Erokhin, Roustem Khazipov, Jordi Vallverd\"{u}, Hu Bin, Philip Moore, Salvatore Distefano, Pei Wang.}
  
  \ecvtitle{2014 -- 2015}{Lecturer at Innopolis University}
  \ecvitem{}{Lecturing Affective computation course from three perspectives: Philosophical (``Model of six'' by Marvin Minsky), Psychological (Wheel of emotions by Plutchik), Neurobiological: (``Cube of emotions'' by Hugo L\"{o}vheim).}

  \ecvtitle{2006 -- 2014}{Software Design Architect at Fujitsu GDC Russia}
  \ecvitem{}{Leading several research projects in: affective computing, machine cognition, code generation automation, natural language processing. Software Design architect in different projects based on: Scala, OpenCog.RelEx, Neo4j, OpenCog.PLN, Stanford Parser, open NARS, MinorThird, Java, EJB, Hibernate, Spring, IBM MQ, Oracle BPEL.}
  
  \ecvsection{Education and training}

  \ecvitem{2014}{Teaching Excellence, Carnegie Mellon University}{}
  \ecvitem{2009}{Software Architecture}{}
  \ecvitem{2008}{Software Requirements Analysis}{}
  \ecvitem{2007}{Managing Software Project Team}{}
  
  \ecvtitlelevel{1996--2000}{PHD degree in Math modelling of Electromagnetic fields in plasma. PhD - Thesis Title: ``Electromagnetic picture of high frequency plasma''}{}
  \ecvitem{}{Kazan State Technological University, Russia}

  \pagebreak
  
  \ecvsection{Personal skills}
  \ecvmothertongue{Russian}
  \ecvlanguageheader
  \ecvlanguage{English}{C2}{C2}{C2}{C2}{C2}
  \ecvlanguagefooter
   
  \ecvblueitem{Communication skills}{
  \begin{ecvitemize}
  \item science-pop: I took part in \href{https://www.youtube.com/watch?v=BLvS7h3kRbo}{\emph{TEDx}}, \href{https://vk.com/video-87488544_171504962}{\emph{Science Slam}} and \href{https://www.youtube.com/watch?v=sLLKxvUEA7E}{\emph{JavaDay}} science-pop events in Kazan, as well as I took part in science popular Russian resource \href{https://postnauka.ru/author/talanov}{\emph{postnauka.ru}} especially interesting dedicated to \href{https://postnauka.ru/faq/58727}{\emph{Marvin Minsky and his role in AI}}, also forbes.ru published an interview with me \href{http://www.forbes.ru/mneniya-column/288097-kak-sozdat-emotsionalnyi-iskusstvennyi-intellekt}{\emph{available here}}.

    \item science-art: In KFU we have started the initiative to trigger young unconventional scientists talents in science art direction.
  \end{ecvitemize}
  }
  
  \ecvblueitem{Organisational / managerial skills}{
  \begin{ecvitemize}
  \item I'm managing several small research teams in KFU in multidisciplinary and breaking through projects in computational neuroscience, electronics, computer science, neuro-rehabilitation domains for 4 years.
  \item I have played role of general chair in HCC-2017 conference in KFU.
  \end{ecvitemize}
  }

  \ecvdigitalcompetence{\ecvProficient}{\ecvProficient}{\ecvProficient}{\ecvProficient}{\ecvProficient}
  
  \ecvblueitem{Computer skills}{Highly qualified computer scientist/software design architect/developer with more than 17 years of software design and development experience in different domains.
    \begin{ecvitemize}
      \item \emph{Neurosimulator} NEST, Neuron
      \item \emph{Reasoner} OpenNARS, OpenCog.PLN, Pellet
      \item \emph{Electrophy} OpenElectrophy, Spike viewer, StimFit
      \item \emph{ML, NLP} Rapid miner, Weka, OpenCog.RelEx, Stanford parser, Gate, MinorThird
      \item \emph{OWL and mind map} Protege, FreeMind, XMind
      \item \emph{Programming} Scala, Python, Java, Prolog, Refal, C++, Haskell, PLSQL, TSQL
  \end{ecvitemize}
  }
  
  \ecvblueitem{Driving licence}{A, B}
  
  \ecvsection{Additional information}
  
  \ecvblueitem{Publications}{Author of 22 papers indexed in Scopus. Scopus h-Index=4, Google scholar h-Index=5}

  \ecvblueitem{Awards}{2005 - Received honorary diploma of the Sun microsystems, Java projects competition, for the MILK - domain specific language for web sites creation.}
  
  \ecvblueitem{Grants}{\emph{Robot Dream} project received funding according to the Russian Government Program of Competitive Growth of Kazan Federal University.

\emph{NeuCogAr} project received funding from subsidy allocated to Kazan Federal University for the state assignment in the sphere of scientific activities.

\emph{Erasmus+} with Birmingham City University we have received in 2016 for the lecturers exchange.}

  \ecvblueitem{Conferences}{2010 -- CEE-SECR, 2013 -- AINL, 2015 -- AOC@AMSTA, 2015 -- AINA, 2015 -- BICA, 2016 -- Fierces on BICA, 2016 -- AOC@AMSTA, 2016-- AGI, 2016 -- BICA, 2017 -- BICA, 2017 -- ICAROB, 2017 -- ICINCO.

    I took part in organisation of special sessions of AOC@AMSTA-2016 and AOC@AMSTA-2015. Took part in organisation of local series of Software engineering, seminars AKSES-2014. I was a general chair of HCC-2017 conference.}
  
  \end{europasscv}

\end{document}
