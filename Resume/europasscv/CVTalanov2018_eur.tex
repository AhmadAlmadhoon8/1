% !TEX encoding = UTF-8
% !TEX program = pdflatex
% !TEX spellcheck = en_GB


\documentclass[english,a4paper]{europasscv}
\usepackage[english]{babel}
\usepackage{url}

\ecvname{Max Talanov}
%\ecvaddress{12 Strawberry Hill, Dublin 8 Éire/Ireland}
\ecvmobile{+7 962 571 8296}
%\ecvtelephone{+353 127 6689}
%\ecvworkphone{+353 999 888 777}
\ecvemail{max.talanov@gmail.com}
\ecvhomepage{scholar.google.com/citations?hl=en\&user=SoUgPioAAAAJ}
\ecvgithubpage{www.github.com/max-talanov}
\ecvlinkedinpage{www.linkedin.com/in/max-talanov-a004aa16/}
\ecvim{Skype}{cosmicdustman}

\ecvdateofbirth{12 April 1974}
\ecvnationality{Russian}
\ecvgender{Male}

\ecvpicture[width=3.8cm]{Talanov_Max_2012}

\date{}

\begin{document}
  \begin{europasscv}

  \ecvpersonalinfo

  \ecvbigitem{Job applied for}{Leading researcher}

  \ecvsection{Research projects}

  \ecvtitle{2014 -- Present}{NeuCogAr}
  \ecvitem{}{Neuro-simulation of bio-plausible emotions (based on works of Hugo L\"{o}vheim) in a computational and robotic systems.}
  \ecvitem{\ecvhighlight{Project site}}{\href{https://github.com/research-team/neucogar}{https://github.com/research-team/neucogar}}
  \ecvitem{\ecvhighlight{Breakthrough}}{\emph{Affective computing} - first time the bio-plausible implementation of psycho-emotional states mapped to computational processes will be demonstrated. We have already demonstrated: ``fear-like'', ``disgust-like'', ``joy-like'' states.}
  \ecvitem{\ecvhighlight{In collaboration with:}} {Jordi Vallverd\'u from Universitat Aut\'onoma de Barcelona.}
  
  \ecvtitle{2016 -- Present}{Memristive spinal cord segment prosthesis}
  \ecvitem{}{The bio-mimetic memristive implementation of a mammalian spinal cord circuits in a electronic system, based on polyaniline memristive neurons implementation capable of inhibition and neuromodulation as a prosthesis.}
  \ecvitem{\ecvhighlight{Project site}}{\href{https://github.com/research-team/memristive-spinal-cord}{https://github.com/research-team/memristive-spinal-cord}}
  \ecvitem{\ecvhighlight{Breakthrough}}{\emph{Computer science}: new type of hardware with new options of self-learning and adaptation real-time is implemented using bio-inspired architectures. \emph{Electronics}: the development of memristive direction could lead to a revolution in IT industry, triggering development of highly effective self-learning devices. \emph{Neuro-rehabilitation}: the adaptive walking pattern generator implemented as electronic schematic is a first step in neuronal prosthesis.}
  \ecvitem{\ecvhighlight{In collaboration with:}} {Victor Erokhin from Universit\'{a} degli studi di Parma, Igor Lavrov from Mayo Clinic.}

  \ecvtitle{2017 -- Present}{Memristive reaction diffusion processor}
  \ecvitem{}{The bio-inspired by a dopamine neuromodulatory system the new generation of memristive processors.}
  \ecvitem{\ecvhighlight{Project site}}{\href{https://github.com/research-team/cellCircuit}{https://github.com/research-team/cellCircuit}}
  \ecvitem{\ecvhighlight{Breakthrough}}{\emph{Computer science}: new type of hardware operating in real-time using two timescales of ...
    with new options of self-learning and adaptation real-time is implemented using bio-inspired architectures. \emph{Electronics}: the development of memristive direction could lead to a revolution in IT industry, triggering development of highly effective self-learning devices. \emph{Neuro-rehabilitation}: the adaptive walking pattern generator implemented as electronic schematic is a first step in neuronal prosthesis.}
  \ecvitem{\ecvhighlight{In collaboration with:}} {Victor Erokhin from Universit\'{a} degli studi di Parma, Andrew Adamatzky from University of West England.}
  
  
  \ecvsection{Work experience}
  
  \ecvtitle{August 2002 -- Present}{Independent consultant}
  \ecvitem{}{National Youth Council of Ireland\newline 3 Montague Street, Dublin 2, D02 V327, Ireland}
  \ecvitem{}{Evaluation of European Commission youth training support measures for youth national agencies and young people}
   
  \ecvtitle{March 2002 -- July 2002}{Internship}
  \ecvitem{}{European Commission, Youth Unit, DG Education and Culture \newline 200, Rue de la Loi, 1049 Brussels (Belgium)}
  \ecvitem{}{
  \begin{ecvitemize}
      \item evaluating youth training programmes and the partnership between the Council of Europe and European Commission
      \item organizing and running a 2 day workshop on non-formal education for Action 5 large scale projects focusing on quality, assessment and recognition
      \item contributing to the steering sroup on training and developing action plans on training for the next 3 years. Working on the Users Guide for training and the support measures
  \end{ecvitemize}
  }
  \ecvitem{}{\ecvhighlight{Business or sector}\quad European institution}
  
  \ecvtitle{Oct 2001 -- Feb 2002}{Researcher / Independent Consultant}
  \ecvitem{}{Council of Europe, Budapest (Hungary)}
  \ecvitem{}{Working in a research team carrying out in-depth qualitative evaluation of the 2 year Advanced Training of Trainers in Europe using participant observations, in-depth interviews and focus groups. Work carried out in training courses in Strasbourg, Slovenia and Budapest.}
  
  
  \ecvsection{Education and training}
  
  \ecvtitlelevel{1997--2001}{PhD - Thesis Title: 'Young People in the Construction of the Virtual University’, Empirical research on e-learning}{ISCED~6}
  \ecvitem{}{Trinity College Dublin, The University of Dublin, Ireland}
  
  \ecvtitle{1993--1997}{Bachelor of Science in Sociology and Psychology}
  \ecvitem{}{Trinity College Dublin, The University of Dublin, Ireland}
  \ecvitem{}{
      \begin{ecvitemize}
	\item sociology of risk
	\item sociology of scientific knowledge / information society
	\item anthropology
	\item E-learning and Psychology
	\item research methods
      \end{ecvitemize}
  }
  
  \pagebreak
  
  \ecvsection{Personal skills}
  \ecvmothertongue{English}
  \ecvlanguageheader
  \ecvlanguage{French}{C1}{C2}{B2}{C1}{C2}
  \ecvlanguagecertificate{Diplôme d'études en langue française (DELF) B1}
  \ecvlastlanguage{German}{A2}{A2}{A2}{A2}{A2}
  \ecvlanguagefooter
   
  \ecvblueitem{Communication skills}{
  \begin{ecvitemize}
    \item team work: I have worked in various types of teams from research teams to national league hockey. For 2 years I coached my university hockey team
    \item mediating skills: I work on the borders between young people, youth trainers, youth policy and researchers, for example running a 3 day workshop at CoE Symposium ``Youth Actor of Social Change'', and my continued work on youth training programmes 
    \item intercultural skills: I am experienced at working in a European dimension such as being a rapporteur at the CoE Budapest ``youth against violence seminar'' and working with refugees.
  \end{ecvitemize}
  }
  
  \ecvblueitem{Organisational / managerial skills}{
  \begin{ecvitemize}
    \item whilst working for a Brussels based refugee NGO ``Convivial'' I organized a ``Civil Dialogue'' between refugees and civil servants at the European Commission 20th June 2002
    \item during my PhD I organised a seminar series on research methods
  \end{ecvitemize}
  }

  \ecvdigitalcompetence{\ecvBasic}{\ecvIndependent}{\ecvProficient}{\ecvIndependent}{\ecvBasic}
  
  \ecvblueitem{Computer skills}{
  \begin{ecvitemize}
    \item competent with most Microsoft Office programmes
    \item experience with HTML
  \end{ecvitemize}
  }
  
  
  \ecvblueitem{Other skills}{Creating pieces of Art and visiting Modern Art galleries. Enjoy all sports particularly hockey, football and running. Love to travel and experience different cultures.}

  \ecvblueitem{Driving licence}{A, B}
  
  \ecvsection{Additional information}
  
  \ecvblueitem{Publications}{\textit{How to do Observations: Borrowing techniques from the Social Sciences to help Participants do Observations in Simulation Exercises}, Coyote EU/CoE Partnership Publication, (2002).
}
  
  \end{europasscv}

\end{document}
