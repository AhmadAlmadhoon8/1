%% start of file `jdoe_classic.tex'.
%% Copyright 2006 Xavier Danaux.
%
% This work may be distributed and/or modified under the
% conditions of the LaTeX Project Public License version 1.3c,
% available at http://www.latex-project.org/lppl/.

\documentclass{moderncv}
\usepackage{url}
%\usepackage{bibtopic}
%\usepackage[round]{natbib}
%\usepackage{multibib}
%\newcites{sel,all}{Selected works, Papers}

\usepackage[style=authoryear]{biblatex}
\bibliography{my_papers}
\DeclareBibliographyCategory{crucial}

% moderncv styles
%\moderncvstyle{casual}       % optional argument are 'nocolor' (black & white cv) and 'roman' (for roman fonts, instead of sans serif fonts)
\moderncvstyle{classic}       % idem

% character encoding
\usepackage[utf8]{inputenc}   % replace by the encoding you are using
%
% personal data (the given example is exhaustive; just give what you want)
\firstname{} \familyname{}
%\title{Highly-creative cognitive researcher}
%\address{Via Ciro Menotti, 15 – 47042, Cesenatico (FC), Italy}  % for classic style
%%\address{Via Ciro Menotti, 15 – 47042, Cesenatico (FC), Italy} % for casual style
\address{Max Talanov}
\phone{+7 962 571 8296} 
\email{max.talanov@gmail.com}
%\extrainfo{\url{https://www.scopus.com/authid/detail.uri?authorId=41762833600}}
%\extrainfo{\url{https://www.researchgate.net/profile/Max_Talanov}}
\extrainfo{\url{https://scholar.google.com/citations?hl=en&user=SoUgPioAAAAJ}}
%\extrainfo{\url{www.linkedin.com/in/max-talanov-a004aa16}}
%\photo[100pt]{Talanov_Max_2012} % also optional, and the optional argument is the height the picture must be resized to

%\quote{Any intelligent fool can make things bigger, more complex, and more violent. It takes a touch of genius -- and a lot of courage -- to move in the opposite direction.}% also optional
%\quote{Video meliora proboque, deteriora sequor (Ovidio, Metamorphosis, VII, 20)}
%\renewcommand{\listsymbol}{{\fontencoding{U}\fontfamily{ding}\selectfont\tiny\symbol{'102}}} % define another symbol to be used in front of the list items

% the ConTeXt symbol
%\def\ConTeXt{%
 % C%
  %\kern-.0333emo%
  %\kern-.0333emn%
  %\kern-.0667em\TeX%
  %\kern-.0333emt}

% slanted small caps (only with roman family; the sans serif font doesn't exists :-()
%\usepackage{slantsc}
%\DeclareFontFamily{T1}{myfont}{}
%\DeclareFontShape{T1}{myfont}{m}{scsl}{ <-> cork-lmssqbo8}{}
%\usefont{T1}{myfont}{m}{scsl}Testing the font

% command and color used in this document, independently from moderncv
%\definecolor{see}{rgb}{0.5,0.5,0.5}% for web links
%\newcommand{\up}[1]{\ensuremath{^\textrm{\scriptsize#1}}}% for text subscripts

%----------------------------------------------------------------------------------
%            content
%----------------------------------------------------------------------------------
6\begin{document}
%\bibliographystyle{alpha}
%\maketitle
%\makequote
\makecvtitle

%\section{Personal Information}
%\cvitem{Name}{\small Max Talanov}
%\cvitem{Passport}{\small Russia}
%\cvitem{Date of birth}{\small 12 April 1974}
%\cvitem{Contacts}{\small mtalanov@it.kfu.ru (preferred), max.talanov@gmail.com}{}
%\cvitem{Research Gate}{https://www.researchgate.net/profile/Max\char`_Talanov}
 
\section{Research interests}

\cvitem{}{\small The sphere of my multidisciplinary interest is currently focused on several cognitive domains including: cyberization of bodies, biohacking, artificial intelligence, psychology, neurobiology and cell biology. Specifically I could identify my interest in the following list of directions: neuromorphic computing, neurosimulations, bio-inspired artificial intelligence, affective computing, computational neurobiology  that are based on cognitive architectures, organic/memristive electronics and robotics.}

\section{Research experience}
   
\cvitem{}{\small Currently I hold a position head of neuromorphic computing and neurosimulations laboratory where we develop several breaking through projects in collaboration with institutions from the US, Italy, France, Finland.
  Till the November of 2017 I was deputy director for science in Information Technology and Intelligent Systems institute (ITIS) of the Kazan Federal University and the projects manager of laboratory of the neurobiology and for 3 years managing three breaking-through projects (see below). Also I have industrial experience as software design architect and team leader for 8 years in international industrial as well as research projects in Fujitsu, where I was leading researcher of R\&D projects in artificial intelligence, natural language processing, probabilistic reasoning, see TU, Menta, IDP projects below.}

\section{Projects}


\cvitem{Memristive spinal cord segment prosthesis (2016 -- now)}
       {The bio-mimetic memristive implementation of a mammalian spinal cord circuits capable of a walking pattern generation in a form of electronic system as a prosthesis for patients with spinal cord trauma, based on memristive neurons implementation capable of inhibition and neuromodulation.
         \emph{In collaboration with}: Victor Erokhin from Universit\'{a} degli studi di Parma, Igor Lavrov from Mayo Clinic.
         \href{https://github.com/research-team/memristive-spinal-cord}{\emph{Project site.}}}
  
\cvitem{Memristive reaction diffusion processor (2017 -- 2018)}{The bio-inspired by a dopamine neuromodulatory system the new generation of memristive processors. \emph{In collaboration with:} Victor Erokhin from Universit\'{a} degli studi di Parma, Andrew Adamatzky from University of West England.
\href{https://github.com/research-team/cellCircuit}{\emph{Project site}}.
  }

\cvitem{Multi-compartment nociceptive ATP signaling (2017 -- now)}{The neuro-simulation project of an ATP signaling and spikes auto-generation in sensory fibers of mammalian cells using neurosimualtory environment Neuron, model of diffusion and different models of ion channels.
  \emph{In collaboration with:} Rashid Ginniatullin from University of Eastern Finland.}
  
\cvitem{NeuCogAr (2014-2018)}{Neurobiologically inspired cognitive architecture for simulation of neurobiologically plausible emotions in a computational and robotic systems based on neural simulations. ``How can we make machine feel emotion'' this is the key question of the project. We have selected the ``cube of emotions'' as the neuro-psychological basement of our reimplementation project. It was published by Hugo L\"{o}vheim in 2012 and maps three monoamine neuromodulators: noradrenaline, serotonin, noradrenaline and basic emotional states (affects): fear, joy, disgust, humiliation, anger, interest, surprise, distress. We have extended the 3D model with the influence over computational processes parameters (computing utilization, computing distribution, memory distribution, storage volume, storage bandwidth) based on role of neuromodulators. \href{https://github.com/research-team/neucogar}{\emph{Project site}}.}

\cvitem{Memristive brain (2016-now)}{The memristive electronics part of the Robot Dream project dedicated to bio-inspired memristive implementation of a mammalian brain circuits and brain areas capable of real-time emotional processing in a robotic system, based on memristive neurons implementation capable of inhibition and neuromodulation. The problem is that the bio-plausible simulation of a mammalian brain is really slow and can not be used for real-time processing, but robotic embodiment should be operating real-time. We have started the Robot Dream project to solve this problem still keeping bio-plausibility in integration with real-time robotic system. A robotic management system could be implemented using several approaches including reimplementation of mammalian brain structures in electronic schematic. \href{https://github.com/research-team/memristive-brain}{\emph{Project site}}.}

\cvitem{Robot Dream (2015-2017)}{\small The integration of a HPC mammalian brain simulation with real-time robotic system. Two phase architecture based on working metaphor of mammalian dream. The emotional experience collection, processing and behavioral strategy update are based on neural simulations in HPC cluster a robotic system is based on memristive implementation of a mammalian brain circuits. The simulated brain with emotional drives needs the embodiment to have an interface with the real world with all the complexity and variety of inbound stimula. We could not put the cluster based simulation that is not real time into the robotic embodiment with the computational capacity similar to one notebook. To solve the computational capacity gap problem we have introduced the two phases approach. The working metaphor is the dreaming and wake phase of a mammalian life.\href{https://github.com/research-team/robot-dream}{\emph{Project site}}.}

\cvitem{QME (2015-2017)}{\small Quantitative model of emotions based on neuromodulatory model of affects or ``cube of emotions'' by Hugo L\"{o}vheim implemented in the spiking neuronal network NEST.}

\cvitem{BioDynaMo (2015-2017)}{\small HPC framework of the bio-plausible dynamic, growing neural tissues with  of a mammalian growing brain.}

\cvitem{GDP (2015-2017)}{\small The bio-plausible simulation of a hippocampus Giant Depolarizing Potentials with bio-plausible mathematical models of neurobiological processes.}

\cvitem{TU (2012-2016)}{\small Thinking-Understanding. The cognitive architecture implementing the approach of the intelligent system for a help desk automation based on: machine cognition: natural language processing (NLP), probabilistic reasoning and ``Model of six'' the model of human mental processes by Marvin Minsky}
%\cvitem{Cell radio (2015-now)}{\small Nanometers neural activity sensors to be placed, powered and radiate radio frequency in neuronal cells}
\cvitem{Menta (2011-2012)}{\small The framework for automatic software application construction via genetic algorithms and NLP}
\cvitem{IDP (2010-2011)}{\small Intellectual document processing, the project for data-mining of unstructured documents via NLP}

%% Breakthrough
\section{Breakthroughs}

\cvitem{Memrstive spinal cord segment prosthesis}{
\emph{Cyberisation of bodies}: first time the bio-compatible part of the nervous system (spinal cord segment) will be re-implemented as memristive electronic schematic reproducing its functionality. \emph{Machine to brain interface}: first time the self-adaptive and self-learning interface to integrate electronic devices with a mammalian nervous system is proposed. \emph{Computer science}: new type of hardware with new options of self-learning and adaptation real-time is implemented using bio-inspired architectures. \emph{Electronics}: the development of memristive direction could lead to a revolution in IT industry, triggering development of highly effective self-learning devices. \emph{Neuro-rehabilitation}: the adaptive walking pattern generator implemented as electronic schematic is a first step in bio-mimetic neuronal prosthesis domain.}

\cvitem{Memristive reaction diffusion processor}{\emph{Computer science}: new type of hardware operating in real-time using three timescales of BZ reaction, memristive STDP learning and electronic signal transmission.}

\cvitem{Multi-compartment nociceptive ATP signaling}{\emph{Neuroscience}: First time the prolonged auto-generation of spikes triggered by ATP will be demonstrated via bio-plausible neuro-simulation.}

\cvitem{NeuCogAr}{\emph{Affective computing} - first time the bio-plausible implementation of psycho-emotional states mapped to computational processes will be demonstrated. \emph{Cognitive architectures and robotics} - first time the bio-plausible emotional drives will be implemented to form behavioral strategies of an artificial system. We have already demonstrated: ``fear-like'' and ``disgust-like'' states.}

\cvitem{Memristive brain}{\emph{Computer science, AI and robotics}: a new level using bio-inspired architectures with understanding of the functions of neurobiological mechanisms implemented in new type of hardware with new options of self-learning and adaptation in real-time. \emph{Electronics}: the development of memristive direction could lead to a revolution in IT industry. \emph{Neuroscience}: the brain-computer interface has new boost with use of memristors as the interface between living cells and not-living electronic memristors in a hybrid system.}

\cvitem{Robot dream}{\emph{Robotics} - first time the bio-plausible emotional driven cognitive architecture integrated with robotics embodiment will be demonstrated with sensory input and motor output neural systems}

\cvitem{QME}{\emph{Breakthrough} First time the 3D model of emotions will be validated on mammals that could create the basement for quantitative approaches in psychology, marketing, economics.}

\section{Outcomes}

\cvitem{NeuCogAr and Memristive brain projects}{\small\emph{Implemented DA, NA, 5HT subsystems}: the prototype is the feasibility study of the monoamines model of emotions and the proof of concept. The prototype must contain the three major parts involved in DA, 5HT and NA neuromodulation implemented as bio-plausible neuro-simulation. Simulation experiments should contain measurements of computational system parameters: computational performance, memory consumption, activation of particular objects in memory. The prototype should demonstrate the mapping of simulated particular psycho-emotional state with computational system parameters.}

\cvitem{}{\small\emph{Monoamine model of emotions prototype}: the integrated prototype and the proof of concept for DA, 5HT and NA neuromodulation subsystems implemented as bio-plausible neuro-simulation. Simulation experiments should contain measurements of computational system parameters: computational performance, memory consumption, activation of particular objects in memory. The prototype should validate H1 along to all three axis and implement all 8 basic emotional states: fear, disgust, humiliation, joy, interest, surprise, anger, distress.}

\cvitem{}{\small\emph{Basic emotional reactions prototype}: the validated and tuned prototype that should demonstrate not only the feasibility but also the neurobiological plausibility with the option for use in the simulation of complex psychological problems.}

\cvitem{}{\small\emph{Inhibitory memristive neuron physical prototype}: the prototype to demonstrate the proof of concept of the physical memristive implementation of the electronic neuron schematic capable of at least one type of inhibitory (“sombrero”)  and excitatory (Hebbian) learning.}

\cvitem{}{\small\emph{Neuromodulatory memristive neuron physical prototype}: the prototype to demonstrate the proof of concept of the physical memristive implementation of the electronic neuron schematic capable of neuromodulatory influence over inhibitory and excitatory learning. The neuromodulation base should consist of three monoamines neuromodulators: DA, 5HT, NA.}

\cvitem{}{\small\emph{Physical prototype of inhibitory and neuromodulatory memristive neuron}: the integrated prototype to demonstrate the proof of concept of the memristive implementation of the electronic neuron schematic capable of both neuromodulation via three monoamines: DA, 5HT, NA, inhibitory and excitatory learning.}

\section{Network}

\cvitem{Researchers}{\small Projects listed above involve researchers from different institutions and different countries: Victor Erokhin from Universit\'{a} degli studi di Parma, Jordi Vallverd\'u from Universitat Aut\'onoma de Barcelona, Salvatore Distefano, University of Messina, Roustem Khazipov from INSERM U901, Igor Lavrov form Mayo Clinic, Robert Lowe form University of Gothenburg, Hu Bin from Lanzhou university, Philip Moore form Lanzhou university, Manuel Mazzara form University of Innopolis.}
\cvitem{Institutions}{CERN, New castle university, Intel, Innopolis University.}

\section{Granted patent(s)}

\small Thinking-Understanding the registration of intellectual property of software product or technology.

\end{document}

