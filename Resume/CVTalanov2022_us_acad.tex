%% start of file `jdoe_classic.tex'.
%% Copyright 2006 Xavier Danaux.
%
% This work may be distributed and/or modified under the
% conditions of the LaTeX Project Public License version 1.3c,
% available at http://www.latex-project.org/lppl/.

\documentclass{moderncv}
\usepackage{url}
%\usepackage{bibtopic}
%\usepackage[round]{natbib}
%\usepackage{multibib}
%\newcites{sel,all}{Selected works, Papers}

\usepackage[style=authoryear]{biblatex}
\bibliography{my_papers}
\DeclareBibliographyCategory{crucial}

% moderncv styles
%\moderncvstyle{casual}       % optional argument are 'nocolor' (black & white cv) and 'roman' (for roman fonts, instead of sans serif fonts)
\moderncvstyle{classic}       % idem

% character encoding
\usepackage[utf8]{inputenc}   % replace by the encoding you are using
%
% personal data (the given example is exhaustive; just give what you want)
\firstname{} \familyname{}
%\title{Highly-creative cognitive researcher}
%\address{Via Ciro Menotti, 15 – 47042, Cesenatico (FC), Italy}  % for classic style
%%\address{Via Ciro Menotti, 15 – 47042, Cesenatico (FC), Italy} % for casual style
\address{Max Talanov}
\phone{+7 962 571 8296}
\email{max.talanov@gmail.com}
%\extrainfo{\url{https://www.scopus.com/authid/detail.uri?authorId=41762833600}}
%\extrainfo{\url{https://www.researchgate.net/profile/Max_Talanov}}
\extrainfo{\url{https://scholar.google.com/citations?hl=en&user=SoUgPioAAAAJ}}
%\extrainfo{\url{www.linkedin.com/in/max-talanov-a004aa16}}
%\photo[100pt]{Talanov_Max_2012} % also optional, and the optional argument is the height the picture must be resized to

%\quote{Any intelligent fool can make things bigger, more complex, and more violent. It takes a touch of genius -- and a lot of courage -- to move in the opposite direction.}% also optional
%\quote{Video meliora proboque, deteriora sequor (Ovidio, Metamorphosis, VII, 20)}
%\renewcommand{\listsymbol}{{\fontencoding{U}\fontfamily{ding}\selectfont\tiny\symbol{'102}}} % define another symbol to be used in front of the list items

% the ConTeXt symbol
%\def\ConTeXt{%
% C%
%\kern-.0333emo%
%\kern-.0333emn%
%\kern-.0667em\TeX%
%\kern-.0333emt}

% slanted small caps (only with roman family; the sans serif font doesn't exists :-()
%\usepackage{slantsc}
%\DeclareFontFamily{T1}{myfont}{}
%\DeclareFontShape{T1}{myfont}{m}{scsl}{ <-> cork-lmssqbo8}{}
%\usefont{T1}{myfont}{m}{scsl}Testing the font

% command and color used in this document, independently from moderncv
%\definecolor{see}{rgb}{0.5,0.5,0.5}% for web links
%\newcommand{\up}[1]{\ensuremath{^\textrm{\scriptsize#1}}}% for text subscripts

%----------------------------------------------------------------------------------
%            content
%----------------------------------------------------------------------------------
\begin{document}
%\bibliographystyle{alpha}
%\maketitle
%\makequote
    \makecvtitle

%\section{Personal Information}
%\cvitem{Name}{\small Max Talanov}
%\cvitem{Passport}{\small Russia}
%\cvitem{Date of birth}{\small 12 April 1974}
%\cvitem{Contacts}{\small mtalanov@it.kfu.ru (preferred), max.talanov@gmail.com}{}
%\cvitem{Research Gate}{https://www.researchgate.net/profile/Max\char`_Talanov}


    \section{Personal Profile}

    \cventry
    {Bio}{Highly-effective manager and the head of the laboratory with broad industrial and research experience}{}{}{}
    {I am a highly productive manager with broad experience in international project management in several countries: the UK, Sweden, Finland, France, for customers including NHS, RCUK, Volvo, Nokia. Currently, I am CRO in the B-Rain startup company with internationally recognized results, also I have been running the Neuromorphic computing and Neurosimulations laboratory for the last six years managing breaking-through projects in augmented AI, robotics, and neuromorphic computing (see below).
    I have wide industrial experience as an R\&D project manager and tech leader and software design architect for eight years in international industrial projects in Fujitsu.}

    \cvitem
    {Strengths}{\small Highly innovative and effective manager of heterogeneous and distributed research and development teams, having a multidisciplinary mindset in computer science, IT, cognitive science, neuroscience, electronics, philosophy.}
%{Weaknesses}{Unable to work for a long time on a trivial tasks, has difficulties to work with demotivated team}

    \section{Education Background}
    \cventry{2000}{PHD degree in Math modeling}{Kazan State Technological University, Russia}{Faculty of enterprise processes management}{}{}{}

    \section{Professional Experience}

    \cventry{current}{CRO}{}{B-Rain labs LLC}{}
    {Management of R\&D projects, establishing research and development strategy, research networking with other companies, international communications, and collaborations. Management distributed heterogeneous team of researchers and developers using agile methodologies, motivation, planning, and budgeting. Breaking-through projects software and hardware R\&D: Neuromorphic neuroprosthesis, Migraine and pain simulation, GRAS - GPU oriented neurosimulator. Management of collaboration with research centers: CNR-IMEM (Italy), NRC “Kurchatov Institute” (Russia), Mayo Clinic (USA), Skolkovo Institute of Science and Technology (Russia), HSE University (Russia); experts: Victor Erokhin, Vyacheslav Demin, Igor Lavrov. }

    \cventry[1em]{current}{Head of neuromorphic computing and neurosimulations laboratory}{}{Kazan Federal University}{}
    {Management of the scientific group; leading multidisciplinary break-through research in neuromorphic computing, affective computing, organic electronics, brain to computer interface and machine to brain interface, neurobiologically inspired systems. Management of grant applications, publication activities; bachelor, master, and Ph.D. students, international communications and collaborations. In collaboration with research centers: Universit\'{a} degli Studi di Parma, University of Eastern Finland, Unconventional computing lab of the University of Western England, Universitat Aut\'onoma de Barcelona, Universit\'{a} degli Studi di Messina, NRC “Kurchatov Institute”, Lobachevsky University, HSE University, Innopolis University; scientists: Rashid Giniatullin, Andrew Adamatzky, Jordi Vallverd\'{u}, Salvatore Distefano, Victor Kazantsev.}

    \cventry[1em]{2018-2019}{Head of computational neurotechnology projects of the neuroscience laboratory}{}{KFU}{}{Management of the scientific group; leading multidisciplinary break-through research in computational neurobiology and management of grant applications, publication activities; managing bachelor and master students; international communications and collaborations. In collaboration with research centers: the University of Eastern Finland, INSERM; scientists: Roustem Khazipov, Rashid Giniatullin.}

    \cventry[1em]{2017-2018}{Deputy director for science of the Information Technology and Intelligent Systems Institute (ITIS)}{}{KFU}{}{Management of research activities: planning, funding, publications strategy, publicity, and international networking management.}

    \cventry[1em]{2014-2015}{Lecturer}{}{Innopolis university}{}{Lecturing Affective computation course from three perspectives: Philosophical (``Model of six'' by Marvin Minsky), Psychological (Wheel of emotions by Plutchik), Neurobiological: (``Cube of emotions'' by Hugo L\"{o}vheim).}

    \cventry[4em]{2006-2014}{Software Design Architect}{}{Fujitsu GDC Russia}{}{Leading R\&D projects the domains of affective computing, machine cognition, machine learning, code generation automation, natural language processing in several countries, including the UK, Sweden, Finland, France.
    Software Design architect in different projects based on Scala, OpenCog.RelEx, Neo4j, OpenCog.PLN, Stanford Parser, open NARS, MinorThird, Java, EJB, Hibernate, Spring, IBM MQ, Oracle BPEL}{}{}

    \section{Projects}
    \cvitem{Neuro-simulation neuroprosthesis (2020-now)}
    {\small \emph{Description}: The augmented AI project for the reimplementation of walking pattern via neurosimulation of spinal cord segment model for patients with complete spinal cord trauma. \newline{}
    \emph{Breakthrough}:
    \emph{Body Cyberisation}: the first reimplementation of the bio-compatible part of the nervous system (central pattern generator) via neurosimulation reproducing its functionality.
    \emph{Machine to brain interface}: the first implementation of a self-adaptive ``speaking nervous system language'' interface to integrate electronic devices with the mammalian nervous system.
    \emph{Robotics}: the reimplementation of a part of a mammalian nervous system part compatible with modern electronics that leads towards standalone intelligent robotic systems operating in real-time.
    \emph{In collaboration with:} Igor Lavrov, Mayo Clinic, USA.}

    \cvitem{Emotional social robot ``Emotico'' project (2020-now)}
    {\small \emph{Description}: The simulation of psycho-emotional states based robotic project for further integration in the social environment. \newline{}
    \emph{Breakthrough}:
    \emph{Robotics}: the first time the robotic system with the influence of dopamine and serotonin was proposed and implemented.
    \emph{In collaboration with:} Evgeni Magid, KFU, Russia.
    \href{https://ieeexplore.ieee.org/document/9073255}{\emph{IEEE paper}}.}

    \cvitem{Migraine and pain simulation (2020-now)}
    {\small \emph{Description}: The neurosimulation of repetitive neuronal activity of A-$\delta$ and C fibers triggering migraine and pain. \newline{}
    \emph{Breakthrough}:
    \emph{Neuroscience}: the first indication of the neurosimulation of ATP mechanisms taking part in pain neuronal activity.
    \emph{In collaboration with:} Rashid Giniatullin, UEF, Finland.
    \href{https://www.frontiersin.org/articles/10.3389/fncel.2020.00135/full}{\emph{Frontiers in Cellular Neuroscience paper 1}}.
    \href{https://www.frontiersin.org/articles/10.3389/fncel.2021.644047/full}{\emph{Frontiers in Cellular Neuroscience paper 2}}.
    }

    \cvitem{Memristive brain (2016-now)}
    {\small \emph{Description}: The project is dedicated to the bio-inspired memristive implementation of mammalian nervous circuits capable of real-time processing in medical or robotic systems. \newline{}
    \emph{Breakthrough}:
    \emph{Computer science, AI and robotics}: the new type of hardware with new options of self-learning and adaptation real-time is implemented using bio-inspired architectures with new level of understanding of the functions of neurobiological mechanisms.
    \emph{Electronics}: the development of memristive direction could lead to a revolution in the IT industry, triggering the development of highly effective self-learning devices.
    \emph{Neuroscience}: the brain-computer interface has a new boost with the use of memristors as the interface between living cells and not-living electronic memristors in a hybrid system.
    \emph{In collaboration with:} Victor Erokhin, CNR-IMEM, Italy.
    \href{https://www.frontiersin.org/articles/10.3389/fnins.2020.00358/full}{\emph{Frontiers in Neuroscience paper}}.}

    \cvitem{GRAS (2020-now)}
    {\small \emph{Description}: The neurosimulation framework and infrastructure development for the distributed high-performance computing of neuronal topologies of several thousands of neurons.
    The design and development of neurosimulator for bio-plausible computing of neural networks of a spinal cord, cortical columns, as well as real-time one-board wearable computers real-time processing of neuronal activity.
    \emph{In collaboration with:} Igor Lavrov, Mayo Clinic, USA.}

    \cvitem{NeuCogAr (2014-2018)}
    {\small \emph{Description}: Neurobiologically inspired cognitive architecture for simulation of neurobiologically plausible emotions (based on works of Hugo L\"{o}vheim) in computational and robotic systems based on neural simulations.
    \emph{Breakthrough}: \emph{Affective computing}: the first time the bio-plausible implementation of psycho-emotional states mapped to computational processes is demonstrated. \emph{Cognitive architectures and robotics}: the first time the bio-plausible emotional drives is implemented to form behavioral strategies of an artificial system. We have already demonstrated: the ``fear-like'' and ``disgust-like'' states.
    \emph{In collaboration with:} Jordi Vallverd\'u, Universitat Aut\'onoma de Barcelona.}

    \cvitem{Robot Dream (2015-2017)}
    {\small \emph{Description}: The integration of an HPC mammalian brain simulation with a real-time robotic system. Two-phase architecture based on the working metaphor of a mammalian dream. The ``dream phase'' consists of emotional experience collection, processing, and behavioral strategy update is implemented as neural simulations on HPC cluster. The ``wake'' robotic system is based on the memristive implementation of mammalian brain circuits, implemented in the memristive brain project.
    \emph{Breakthrough}: \emph{Robotics}: the first time the bio-plausible emotional driven cognitive architecture integrated with robotics embodiment will be demonstrated including sensory input and motor output neural systems.}

    \cvitem{BioDynaMo (2015-2017)}
    {\small \emph{Description}: HPC framework of the bio-plausible dynamic, growing neural tissues simulations including a mammalian growing brain.
    \emph{In collaboration with:} CERN, Newcastle university, Intel, Innopolis University.}

%extend
    \cvitem{TU (2012-2017)}
    {\small \emph{Description}: Thinking-Understanding. The cognitive architecture implementing the approach of the intelligent system for helpdesk automation based on machine cognition: natural language processing (NLP), probabilistic reasoning and ``Model of six'' - the model of human mental processes by Marvin Minsky}

    \cvitem{Menta (2011-2012)}
    {\small \emph{Description}: The framework for automatic software application development via genetic algorithms and NLP}

    \cvitem{IDP (2010-2011)}
    {\small \emph{Description}: Intellectual document processing, the project for data-mining of unstructured documents via NLP and ML}

    \section{Professional Training}
    \cvitem{2014}{\small Teaching Excellence, Carnegie Mellon University, USA}
    \cvitem{2009}{\small Software Architecture, Carnegie Mellon University, USA}
    \cvitem{2008}{\small Software Requirements Analysis, Carnegie Mellon University, USA}
    \cvitem{2007}{\small Managing Software Project Team, Carnegie Mellon University, USA}

    \section{Teaching Experience}
    \cvitem{2017-Now, KFU}{\small \emph{Artificial intelligence}: the introduction in artificial intelligence including: machine learning and reinforcement learning, decision making, reasoning, knowledge bases and data representations, natural language processing, intelligent agents. During the course students should develop the AI project using principles and technologies discussed in the course.}

    \cvitem{2016-2017, KFU; 2014-2015, Innopolis University}{\small \emph{Affective computation}: the extended view on the emotions and reimplementation in a computational system problem including: philosophical, psychological, neurobiological and computational perspectives. The course starts from birds eye view on the problems, carries on to neurobiological details of a neuromodulation and psychological models of emotions, then progresses into philosophical questions of consciousness and thinking and ends up with cognitive architectures and spiking neural networks review. \href{https://github.com/max-talanov/1/blob/master/affective_computing_course/syllabus.md}{\emph{Course syllabus online}}.}

    \cvitem{2014-2016, KFU; 2014-2015, Innopolis University}{\small \emph{Software Design Architecture}: the course for bachelor students, intended to be starting point from basics of software design and UML to principles and design patterns, with extended use of practical examples. During the course students should develop project using principles of design studied. \href{https://github.com/max-talanov/1/blob/master/software_design_course/plan.md}{\emph{Course syllabus online}}.}

    \section{Grants}
    %\cvitem{}{\small \emph{Neurosimulation neuroprosthesis} project received funding from the Assistance for innovation Fund.}
    %\cvitem{}{\small \emph{Memristive spinal cord segment prosthesis} project received funding from the Russian Basic Research Fund.}
    %\cvitem{}{\small \emph{Robot Dream} project received funding according to the Russian Government Program of Competitive Growth of Kazan Federal University.}
    %\cvitem{}{\small \emph{NeuCogAr} project received funding from subsidy allocated to Kazan Federal University for the state assignment in the sphere of scientific activities.}
    %\cvitem{}{\small \emph{Erasmus+} with Birmingham City University we have received in 2016 and 2018 for the lecturers exchange.}
    \emph{Neurosimulation neuroprosthesis} project received funding from the Assistance for Innovation Fund. \\
    \emph{Memristive spinal cord segment prosthesis} project received funding from the Russian Basic Research Fund. \\
    \emph{Robot Dream} project received funding according to the Russian Government Program of Competitive Growth of Kazan Federal University. \\
    \emph{NeuCogAr} project received funding from subsidy allocated to Kazan Federal University for the state assignment in the sphere of scientific activities. \\
    \emph{Erasmus+} with Birmingham City University we have received in 2016 and 2018 for the lecturers exchange.


    \cvitem{ }{ }
    
    \section{Prizes and Awards}
    %\cvitem{2018}{\small Best poster award in ESCI-2018 conference.}
    %\cvitem{2017}{\small Best paper award in DESE-2017 conference.}
    %\cvitem{2005}{\small Honorary diploma of the Sun microsystems, Java projects competition, for the MILK - domain specific language for web sites creation}
    2021 - Our article ``Modeling a Nociceptive Neuro-Immune Synapse Activated by ATP and 5-HT in Meninges: Novel Clues on Transduction of Chemical Signals Into Persistent or Rhythmic Neuronal Firing'' was selected to \href{https://www.frontiersin.org/research-topics/21434/cellular-neurophysiology-editors-pick-2021}{Cellular Neurophysiology Editors' Pick 2021 collection}.\\
    2018 - Best poster award in ESCI-2018 conference.\\
    2017 - Best paper award in DESE-2017 conference.\\
    2005 - Received honorary diploma of the Sun microsystems, Java projects competition, for the MILK - domain specific language for web sites creation.

    \section{Patents}
    \emph{Recovery of sensorimotor with neuromorphic system and method thereof} international patent pending.\\
    \emph{Simulator of neuronal activity based on spiking neural network GRAS} the registration of intellectual property of software product or technology N~2020663801.
    \emph{Thinking-Understanding} the registration of intellectual property of software product or technology N~2016618910.\\
    
    
    \section{Papers}
    Total 45 published papers at the moment, H-index 11(Google scholar).
    In 2020 and 2021 my scientific group published papers in world leading scientific journals:
    Frontiers in Computer Science, Frontiers in Neuroscience, Frontiers in Cellular Neuroscience, Nature Scientific Reports.

%\addtocategory{crucial}{talanov_2015,vallverdu_2016a,jordi2016importance}
%\printbibliography[title={Selected papers}, category = crucial]

% all papers (my_papers.bib)
\nocite{*}
\printbibliography[title={Papers}]

    \section{Conference and Workshop Activities}
    2010 -- CEE-SECR, 2013 -- AINL, 2015 -- AOC@AMSTA, 2015 -- AINA, 2015 -- BICA, 2016 -- Fierces on BICA, 2016 -- AOC@AMSTA, 2016-- AGI, 2016 -- BICA, 2017 -- BICA, 2017 -- ICAROB, 2017 -- ICINCO, 2017 -- DESE, 2018 -- ESCI, 2018 -- Volga neuroscience meeting, 2021 -- IROS, 2021 -- BICA, 2021 -- BF-NAICS.

    I took part in organisation of special sessions of AOC@AMSTA-2016 and AOC@AMSTA-2015 and local series of Software engineering seminars AKSES-2014. I was a general chair of HCC-2017 conference.

    \section{Science-pop activities}
    I had science-pop lecture ``Cyberpunk revolution'' in cultural center ``Smena'' in 2021.
    I took part in \href{https://www.youtube.com/watch?v=BLvS7h3kRbo}{\emph{TEDx}}, \href{https://vk.com/video-87488544_171504962}{\emph{Science Slam}} and \href{https://www.youtube.com/watch?v=sLLKxvUEA7E}{\emph{JavaDay}} science-pop events in Kazan, as well as I took part in science popular Russian resource \href{https://postnauka.ru/author/talanov}{\emph{postnauka.ru}} especially interesting paper is dedicated to \href{https://postnauka.ru/faq/58727}{\emph{Marvin Minsky and his role in AI}}, also forbes.ru published an interview with me \href{http://www.forbes.ru/mneniya-column/288097-kak-sozdat-emotsionalnyi-iskusstvennyi-intellekt}{\emph{available here}}.

    \section{Language Competence}
    \cvitem{English}{Full professional proficiency}
    \cvitem{Russian}{Mother tongue}

%\section{References}

%Dr. Igor Lavrov, Mayo Clinic, igor.lavrov@gmail.com.\\
%Prof. Roustem Khazipov, INSERM, roustem.khazipov@inserm.fr, +33491828141\\
%Prof. Victor Erokhin, CNR-IMEM, victor.erokhin@fis.unipr.it, +393281019272\\
%Dr. Salvatore Distefano, Messina University, salvatdi@gmail.com, +390903977318

\end{document}

