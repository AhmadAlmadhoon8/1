%% start of file `jdoe_classic.tex'.
%% Copyright 2006 Xavier Danaux.
%
% This work may be distributed and/or modified under the
% conditions of the LaTeX Project Public License version 1.3c,
% available at http://www.latex-project.org/lppl/.

\documentclass{moderncv}
\usepackage{url}
%\usepackage{bibtopic}
%\usepackage[round]{natbib}
%\usepackage{multibib}
%\newcites{sel,all}{Selected works, Papers}

\usepackage[style=authoryear]{biblatex}
\bibliography{my_papers}
\DeclareBibliographyCategory{crucial}

% moderncv styles
%\moderncvstyle{casual}       % optional argument are 'nocolor' (black & white cv) and 'roman' (for roman fonts, instead of sans serif fonts)
\moderncvstyle{classic}       % idem

% character encoding
\usepackage[utf8]{inputenc}   % replace by the encoding you are using
%
% personal data (the given example is exhaustive; just give what you want)
\firstname{} \familyname{}
%\title{Highly-creative cognitive researcher}
%\address{Via Ciro Menotti, 15 – 47042, Cesenatico (FC), Italy}  % for classic style
%%\address{Via Ciro Menotti, 15 – 47042, Cesenatico (FC), Italy} % for casual style
\address{Max Talanov}
\phone{+7 962 571 8296} 
\email{max.talanov@gmail.com}
%\extrainfo{\url{https://www.scopus.com/authid/detail.uri?authorId=41762833600}}
%\extrainfo{\url{https://www.researchgate.net/profile/Max_Talanov}}
\extrainfo{\url{https://scholar.google.com/citations?hl=en&user=SoUgPioAAAAJ}}
%\extrainfo{\url{www.linkedin.com/in/max-talanov-a004aa16}}
%\photo[100pt]{Talanov_Max_2012} % also optional, and the optional argument is the height the picture must be resized to

%\quote{Any intelligent fool can make things bigger, more complex, and more violent. It takes a touch of genius -- and a lot of courage -- to move in the opposite direction.}% also optional
%\quote{Video meliora proboque, deteriora sequor (Ovidio, Metamorphosis, VII, 20)}
%\renewcommand{\listsymbol}{{\fontencoding{U}\fontfamily{ding}\selectfont\tiny\symbol{'102}}} % define another symbol to be used in front of the list items

% the ConTeXt symbol
%\def\ConTeXt{%
 % C%
  %\kern-.0333emo%
  %\kern-.0333emn%
  %\kern-.0667em\TeX%
  %\kern-.0333emt}

% slanted small caps (only with roman family; the sans serif font doesn't exists :-()
%\usepackage{slantsc}
%\DeclareFontFamily{T1}{myfont}{}
%\DeclareFontShape{T1}{myfont}{m}{scsl}{ <-> cork-lmssqbo8}{}
%\usefont{T1}{myfont}{m}{scsl}Testing the font

% command and color used in this document, independently from moderncv
%\definecolor{see}{rgb}{0.5,0.5,0.5}% for web links
%\newcommand{\up}[1]{\ensuremath{^\textrm{\scriptsize#1}}}% for text subscripts

%----------------------------------------------------------------------------------
%            content
%----------------------------------------------------------------------------------
\begin{document}
%\bibliographystyle{alpha}
%\maketitle
%\makequote
\makecvtitle

%\section{Personal Information}
%\cvitem{Name}{\small Max Talanov}
%\cvitem{Passport}{\small Russia}
%\cvitem{Date of birth}{\small 12 April 1974}
%\cvitem{Contacts}{\small mtalanov@it.kfu.ru (preferred), max.talanov@gmail.com}{}
%\cvitem{Research Gate}{https://www.researchgate.net/profile/Max\char`_Talanov}
 
\section{Formal educational training}

\cvitem{2014}{\small Teaching Excellence, Carnegie Mellon University}
\cvitem{2009}{\small SEP Software Architecture}
\cvitem{2008}{\small SEP Software Requirements Analysis}
\cvitem{2007}{\small SEP Managing Software Project Team}
\cvitem{Conference and Workshop}{\small 2010 -- CEE-SECR, 2013 -- AINL, 2015 -- AOC@AMSTA, 2015 -- AINA, 2015 -- BICA, 2016 -- Fierces on BICA, 2016 -- AOC@AMSTA, 2016-- AGI, 2016 -- BICA, 2017 -- BICA, 2017 -- ICAROB, 2017 -- ICINCO. I took part in organisation of special sessions of AOC@AMSTA-2016 and AOC@AMSTA-2015. Took part in organisation of local series of Software engineering, seminars AKSES-2014.}

\section{Administrative tasks relating to education}

\cvitem{}{\small Currently I have the position of the deputy director for science of the Information Technology and Information Systems institute (ITIS) of the Kazan Federal University, where I manage the research, grants and publication policies. For two years I was hearing two chairs in ITIS the Software engineering chair and Intellectual robotics chair during this period I have established the close collaboration with industrial companies to provide relevant and modern software expertise for students on highly demanded specializations. I take part in the academic committee of the KFU as well as ITIS institute.}

\section{Experience of study programmes, supervision and examinations}

\cvitem{}{\small While managing the Intelligent robotics chair I have built the research based department where students are involved in multidisciplinary research projects from the first day of their involvement in labs of the chair. This project training provided highly effective opportunity for students to start their scientific carrier. Majority of my students have their papers published in Scopus/Web of Science one of them has \href{https://www.scopus.com/authid/detail.uri?authorId=56150559300}{\emph{h-index of 1.}} I have supervised thesis works of the students of my lab and the most significant result was the papers published with the involvement of the students during their thesis works presented in international conferences and journals. I have created and done the lecturing as well as supervision of the project for the following courses:}

\cvitem{2017-Now, Kazan Federal University}{\small \emph{Artificial intelligence}: the introduction in artificial intelligence including: machine learning and reinforcement learning, decision making, reasoning, knowledge bases and representations, natural language processing, intelligent agents. During the course students should develop the AI project using principles and technologies discussed in the course.}

\cvitem{2016-Now, Kazan Federal University; 2014-2015, Innopolis University}{\small \emph{Affective computation}: the extended view on the emotions and reimplementation in a computational system problem  including: philosophical, psychological, neurobiological and computational perspectives. The course starts from birds eye view on the problem, carries on to neurobiological details of a neuromodulation and psychological models of emotions, then progresses into philosophical questions of consciousness and thinking and ends up with cognitive architectures and spiking neural networks review. \href{https://github.com/max-talanov/1/blob/master/affective_computing_course/plan.md}{\emph{Course syllabus online}}.}

\cvitem{2014-2016, Kazan Federal University}{\small \emph{Software Design Architecture}: the course for bachelor students, intended to be starting point from basics of software design and UML to principles and design patterns, with extended use of practical examples. During the course students should develop project using principles of design studied. \href{https://github.com/max-talanov/1/blob/master/software_design_course/plan.md}{\emph{Course syllabus online}}.}

\section{Methods, materials and tools}

\cvitem{}{I'm the head of the ``machine cognition'' laboratory and I use ``train while you do'' approach for the education heavily. To enroll to the laboratory none has to pass the examination test of two parts: fundamentals of neuroscience and programming basics in Pyhton. The test is used for two purposes: filter the best, identify the knowledge level of a student. If a student is accepted in the laboratory, student could select the research project of her/his preferences from the list of the projects of the laboratory. Later the head of the laboratory assigns the supervisor (leading researcher of the project). Supervisor does the micromanagement of the student. We have mandatory weekly updates where we do retrospective and discuss problems of students and assign new tasks for a next week. During updates we also discuss publishing options as well as conferences to attend in case the team of the project achieves significant result. If the student takes significant part in the scientific results of a project we usually provide the option to present result on the international conference. A \emph{thesis work} is special activity that requires concentrated work of a student in the laboratory for a half a year. During this period I insist on heavy participation in the laboratory projects every day with attentive monitoring of the progress of the thesis projects. We usually plan the phases of thesis projects as well as process creation of a thesis work itself and presentation. We usually do the rehearsal of thesis presentation 3-5 times to guarantee the result of thesis presentation. For example in 2017 4 from 5 students that done the thesis works in my laboratory were assessed with highest marks. The courses marking and tracking is done via Moodle system that is closely connected to the course syllabus, that is available online. 
}

\section{Teaching Experience}
  
\cvitem{5. Educational development and applied research into teaching at university, including educational awards}{Descriptions of Participation in educational development projects, specifying any allocated project funds. Indicate the grounds of the project, and its results. Documentation of development projects in the form of reports and articles as well as posters and presentations at meetings and conferences Educational (applied) research projects Educational research training}

\cvitem{6. Reflections on your own teaching practice and future development including student evaluations}{A summary of the main features of your previous teaching practice as well as thoughts on your own future development in relation to teaching responsibilities ahead Reflections on how you have developed your teaching practice based on student evaluations and in interaction with students and colleagues Other university teaching issues that you consider important}

\section{Computing Professional Experience}

\cventry{current}{Deputy director for science of the Information Technology and Information Systems institute (ITIS) in the Kazan Federal University}{}{Kazan, Russian Federation}{}{Management of research activities: planning, funding, publications strategy, publicity and international networking management.}

\cventry{current}{Head of computational neuoro-technology projects of neuroscience laboratory}{}{Kazan, Russian Federation}{}{Management of the scientific group; leading multidisciplinary breakthrough research in: affective computing, neurobiologically inspired systems, computational neurobiology; grant applications, publication activities; managing bachelor and master students, international communications and collaborations, joint laboratory activities. In collaboration with research centers: CERN OpenLab V, Lanzhou University School of information science and engineering, Samsung Research Center, Universit\'{a} degli studi di Parma, Universitat Aut\'onoma de Barcelona; scientists: Victor Erokhin, Roustem Khazipov, Jordi Vallverd\"{u}, Hu Bin, Philip Moore, Salvatore Distefano, Pei Wang.}

\cventry{2014-2015}{Lecturer at Innopolis University}{}{Kazan, Russian Federation}{}{Lecturing Affective computation course from three perspectives: Philosophical (Model of six by Marvin Minsky), Psychological (Wheel of emotions by Plutchik), Neurobiological: (``Cube of emotions'' by Hugo L\"{o}vheim).}

\cventry{2006-2014}{Software Design Architect in Fujitsu GDC Russia}{}{}{}{Leading several research projects: affective computing, machine cognition, code generation automation, natural language processing. Software Design architect in different projects based on: Scala, OpenCog.RelEx, 
Neo4j, OpenCog.PLN, Stanford Parser, open NARS, MinorThird, Java, EJB, Hibernate, Spring, IBM MQ, Oracle BPEL}{}{}

\section{Prizes and Awards}

2005 - Received honorary diploma of the Sun microsystems, Java projects competition, for the MILK - domain specific language for web sites creation.

\section{Grants}

\emph{Robot Dream and Memristive brain} projects received funding according to the Russian Government Program of Competitive Growth of Kazan Federal University.

\emph{NeuCogAr} project received funding from subsidy allocated to Kazan Federal University for the state assignment in the sphere of scientific activities.

\emph{Erasmus+} with Birmingham city university we have received in 2016 for the lecturers exchange. 

\section{Granted patent(s)}

Thinking-Understanding the registration of intellectual property of software product or technology.

%Selected papers
%\addtocategory{crucial}{talanov_2015,vallverdu_2016a,jordi2016importance}
%\printbibliography[title={Most popular papers}, category = crucial]
 
% all papers (my_papers.bib)
\nocite{*}
\printbibliography[title={Papers}]

\cvitem{}{}


\section{Science-pop activities}

I took part in \href{https://www.youtube.com/watch?v=BLvS7h3kRbo}{\emph{TEDx}}, \href{https://vk.com/video-87488544_171504962}{\emph{Science Slam}} and \href{https://www.youtube.com/watch?v=sLLKxvUEA7E}{\emph{JavaDay}} science-pop events in Kazan, as well as I took part in science popular Russian resource \href{https://postnauka.ru/author/talanov}{\emph{postnauka.ru}} especially interesting dedicated to \href{https://postnauka.ru/faq/58727}{\emph{Marvin Minsky and his role in AI}}, also forbes.ru published an interview with me \href{http://www.forbes.ru/mneniya-column/288097-kak-sozdat-emotsionalnyi-iskusstvennyi-intellekt}{\emph{available here}}.

\section{Computing Knowledge}

\cvitem{Neurosimulator}{NEST, Neuron}
\cvitem{Reasoner}{open NARS, OpenCog.PLN, Pellet}
\cvitem{Electrophy}{OpenElectrophy, Spike viewer, StimFit} 
\cvitem{ML, NLP}{Rapid miner, Weka, OpenCog.RelEx, Stanford parser, Gate, MinorThird}
\cvitem{OWL and mind map}{Protege, FreeMind, XMind}
\cvitem{Programming}{Scala, Python, Java, Prolog, Refal, C++, Haskell, PLSQL, TSQL, Java SE, EJB, Hibernate, Spring, Seam, IBM MQ, Oracle BPEL, JBoss, Tomcat, Web Services}
\cvitem{Databases}{MSSQL, Oracle DB, PostgreSQL, Neo4j, HypergraphDB}
\cvitem{Development}{Jetbrains PyCharm, Jetbrains Idea, Eclipse, Visual Studio, Maven, Emacs, ArgoUML, Visual Paradigm, Rational Rose, Enterprise Architect, QTCreator, Oracle JDeveloper} 

\cvitem{OS}{Microsoft Windows, Linux, Mac OS}

\section{Education Background}

\cventry{2000}{PHD degree in Math modelling of Electromagnetic fields in plasma}{Kazan State Technological University, Russia}{Faculty of enterprise processes management}{}{Thesis: \emph{Electromagnetic picture of high frequency plasma}}{}

\section{Language Competence}

\cvitem{English}{Full professional proficiency}
\cvitem{Russian}{Mother tongue}

\end{document}

