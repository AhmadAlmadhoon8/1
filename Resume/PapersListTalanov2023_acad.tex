
%% start of file `jdoe_classic.tex'.
%% Copyright 2006 Xavier Danaux.
%
% This work may be distributed and/or modified under the
% conditions of the LaTeX Project Public License version 1.3c,
% available at http://www.latex-project.org/lppl/.

\documentclass{moderncv}
\usepackage{url}
%\usepackage{bibtopic}
%\usepackage[round]{natbib}
%\usepackage{multibib}
%\newcites{sel,all}{Selected works, Papers}
%style=authoryear

\usepackage[maxcitenames=4, mincitenames=4, maxbibnames=99, minbibnames=99]{biblatex}
\bibliography{my_papers}
\DeclareBibliographyCategory{crucial}

% moderncv styles
%\moderncvstyle{casual}       % optional argument are 'nocolor' (black & white cv) and 'roman' (for roman fonts, instead of sans serif fonts)
\moderncvstyle{classic}       % idem

% character encoding
\usepackage[utf8]{inputenc}   % replace by the encoding you are using
%
% personal data (the given example is exhaustive; just give what you want)
\firstname{} \familyname{}
%\title{Highly-creative cognitive researcher}
%\address{Via Ciro Menotti, 15 – 47042, Cesenatico (FC), Italy}  % for classic style
%%\address{Via Ciro Menotti, 15 – 47042, Cesenatico (FC), Italy} % for casual style
\address{Max Talanov}
\phone{+381 64 519 2005}
\email{max.talanov@gmail.com}
%\extrainfo{\url{https://www.scopus.com/authid/detail.uri?authorId=41762833600}}
%\extrainfo{\url{https://www.researchgate.net/profile/Max_Talanov}}
\extrainfo{\url{https://scholar.google.com/citations?hl=en&user=SoUgPioAAAAJ}}
%\extrainfo{\url{www.linkedin.com/in/max-talanov-a004aa16}}
%\photo[100pt]{Talanov_Max_2012} % also optional, and the optional argument is the height the picture must be resized to

%\quote{Any intelligent fool can make things bigger, more complex, and more violent. It takes a touch of genius -- and a lot of courage -- to move in the opposite direction.}% also optional
%\quote{Video meliora proboque, deteriora sequor (Ovidio, Metamorphosis, VII, 20)}
%\renewcommand{\listsymbol}{{\fontencoding{U}\fontfamily{ding}\selectfont\tiny\symbol{'102}}} % define another symbol to be used in front of the list items

% the ConTeXt symbol
%\def\ConTeXt{%
% C%
%\kern-.0333emo%
%\kern-.0333emn%
%\kern-.0667em\TeX%
%\kern-.0333emt}

% slanted small caps (only with roman family; the sans serif font doesn't exists :-()
%\usepackage{slantsc}
%\DeclareFontFamily{T1}{myfont}{}
%\DeclareFontShape{T1}{myfont}{m}{scsl}{ <-> cork-lmssqbo8}{}
%\usefont{T1}{myfont}{m}{scsl}Testing the font

% command and color used in this document, independently from moderncv
%\definecolor{see}{rgb}{0.5,0.5,0.5}% for web links
%\newcommand{\up}[1]{\ensuremath{^\textrm{\scriptsize#1}}}% for text subscripts

%----------------------------------------------------------------------------------
%            content
%----------------------------------------------------------------------------------
\begin{document}
%\bibliographystyle{alpha}
%\maketitle
%\makequote
    \makecvtitle

%\section{Personal Information}
%\cvitem{Name}{\small Max Talanov}
%\cvitem{Passport}{\small Russia}
%\cvitem{Date of birth}{\small 12 April 1974}
%\cvitem{Contacts}{\small mtalanov@it.kfu.ru (preferred), max.talanov@gmail.com}{}
%\cvitem{Research Gate}{https://www.researchgate.net/profile/Max\char`_Talanov}
    
        \section{Papers}
    Total more than 60 published papers at the moment, H-index 11(Google scholar).
    In 2020 and 2021 my scientific group published papers in world leading scientific journals:
    Frontiers in Computer Science, Frontiers in Neuroscience, Frontiers in Cellular Neuroscience, Nature Scientific Reports.

%\addtocategory{crucial}{talanov_2015,vallverdu_2016a,jordi2016importance}
%\printbibliography[title={Selected papers}, category = crucial]

% all papers (my_papers.bib)
\nocite{*}
\printbibliography[title={All papers}]

    \section{Conference and Workshop Activities}
    2010 -- CEE-SECR, 2013 -- AINL, 2015 -- AOC@AMSTA, 2015 -- AINA, 2015 -- BICA, 2016 -- Fierces on BICA, 2016 -- AOC@AMSTA, 2016-- AGI, 2016 -- BICA, 2017 -- BICA, 2017 -- ICAROB, 2017 -- ICINCO, 2017 -- DESE, 2018 -- ESCI, 2018 -- Volga neuroscience meeting, 2021 -- IROS, 2021 -- BICA, 2021 -- BF-NAICS.

    I took part in organisation of special sessions of AOC@AMSTA-2016 and AOC@AMSTA-2015 and local series of Software engineering seminars AKSES-2014. I was a general chair of HCC-2017 conference.

    \section{Science-pop activities}
    I had science-pop lecture ``Cyberpunk revolution'' in cultural center ``Smena'' in 2021.
    I took part in \href{https://www.youtube.com/watch?v=BLvS7h3kRbo}{\emph{TEDx}}, \href{https://vk.com/video-87488544_171504962}{\emph{Science Slam}} and \href{https://www.youtube.com/watch?v=sLLKxvUEA7E}{\emph{JavaDay}} science-pop events in Kazan, as well as I took part in science popular Russian resource \href{https://postnauka.ru/author/talanov}{\emph{postnauka.ru}} especially interesting paper is dedicated to \href{https://postnauka.ru/faq/58727}{\emph{Marvin Minsky and his role in AI}}, also forbes.ru published an interview with me \href{http://www.forbes.ru/mneniya-column/288097-kak-sozdat-emotsionalnyi-iskusstvennyi-intellekt}{\emph{available here}}.

%\section{References}

%Dr. Igor Lavrov, Mayo Clinic, igor.lavrov@gmail.com.\\
%Prof. Roustem Khazipov, INSERM, roustem.khazipov@inserm.fr, +33491828141\\
%Prof. Victor Erokhin, CNR-IMEM, victor.erokhin@fis.unipr.it, +393281019272\\
%Dr. Salvatore Distefano, Messina University, salvatdi@gmail.com, +390903977318

\end{document}

