%% start of file `jdoe_classic.tex'.
%% Copyright 2006 Xavier Danaux.
%
% This work may be distributed and/or modified under the
% conditions of the LaTeX Project Public License version 1.3c,
% available at http://www.latex-project.org/lppl/.

\documentclass{moderncv}
\usepackage{url}

% moderncv styles
%\moderncvstyle{casual}       % optional argument are 'nocolor' (black & white cv) and 'roman' (for roman fonts, instead of sans serif fonts)
\moderncvstyle{classic}       % idem

% character encoding
\usepackage[utf8]{inputenc}   % replace by the encoding you are using
%
% personal data (the given example is exhaustive; just give what you want)
\firstname{} \familyname{}
%\title{Highly-creative cognitive researcher}
%\address{Via Ciro Menotti, 15 – 47042, Cesenatico (FC), Italy}  % for classic style
%%\address{Via Ciro Menotti, 15 – 47042, Cesenatico (FC), Italy} % for casual style
\address{Max Talanov}
\phone{+7 987 008 3791} 
\email{max.talanov@gmail.com; mtalanov@it.kfu.ru}
\extrainfo{\url{https://www.researchgate.net/profile/Max_Talanov}}
\photo[100pt]{Talanov_Max_2012} % also optional, and the optional argument is the height the picture must be resized to

%\quote{Any intelligent fool can make things bigger, more complex, and more violent. It takes a touch of genius -- and a lot of courage -- to move in the opposite direction.}% also optional
%\quote{Video meliora proboque, deteriora sequor (Ovidio, Metamorphosis, VII, 20)}
%\renewcommand{\listsymbol}{{\fontencoding{U}\fontfamily{ding}\selectfont\tiny\symbol{'102}}} % define another symbol to be used in front of the list items

% the ConTeXt symbol
%\def\ConTeXt{%
 % C%
  %\kern-.0333emo%
  %\kern-.0333emn%
  %\kern-.0667em\TeX%
  %\kern-.0333emt}

% slanted small caps (only with roman family; the sans serif font doesn't exists :-()
%\usepackage{slantsc}
%\DeclareFontFamily{T1}{myfont}{}
%\DeclareFontShape{T1}{myfont}{m}{scsl}{ <-> cork-lmssqbo8}{}
%\usefont{T1}{myfont}{m}{scsl}Testing the font

% command and color used in this document, independently from moderncv
%\definecolor{see}{rgb}{0.5,0.5,0.5}% for web links
%\newcommand{\up}[1]{\ensuremath{^\textrm{\scriptsize#1}}}% for text subscripts

%----------------------------------------------------------------------------------
%            content
%----------------------------------------------------------------------------------
\begin{document}
%\maketitle
%\makequote
\makecvtitle

%\section{Personal Information}
%\cvitem{Name}{\small Max Talanov}
%\cvitem{Passport}{\small Russia}
%\cvitem{Date of birth}{\small 12 April 1974}
%\cvitem{Contacts}{\small mtalanov@it.kfu.ru (preferred), max.talanov@gmail.com}{}
%\cvitem{Research Gate}{https://www.researchgate.net/profile/Max\char`_Talanov}
 
\section{Personal Profile}

\cventry
{Bio}{Highly-creative cognitive researcher}{}{}{}{Has experience in: affective computing, computational neurobiology, brain simulations, machine cognition, natural language processing, probabilistic reasoning. Currently has the position of the head of the Intellectual Robotics department of the Information Technology and Information Systems institute (ITIS) of the Kazan Federal University, where he runs cross-disciplinary projects in: simulation of emotions, human-robot interface, bio-electronics, brain simulation framework, machine cognition and natural language processing. Has industrial experience as software architect and team leader for 16 years in international projects in Fujitsu.}

\cvcomputer
{Strengths}{Highly creative, ideas generator, good in multy-disciplinary research, has cross-disciplinary understanding in: cognitive science, psychology, neuroscience, philosophy, computer science, IT, electronics}
{Weaknesses}{Unable to work for a long time on a trivial tasks, has difficulties to work with demotivated team}

\section{Computing Professional Experience}

\cventry{current}{Acting head of the Intellectual Robotics department in Kazan Federal University}{}{Kazan, Russian Federation}{}{Management of several research labs, scientific projects; planning, funding, publications strategy, publicity and international activity management}

\cventry{current}{Head of Machine Cognition lab}{}{Kazan, Russian Federation}{}{Creation and management of the scientific group; leading research in several directions: affective computation, neurobiologically inspired systems, computational neurobiology; grant applications, publication activities, managing master students, international communications and collaborations, joint lab activities}

\cventry{2015}{Lecturer in Innopolis University}{}{Kazan, Russian Federation}{}{Lecturing Affective computation course from three perspectives: Philosophical (Model of six by Marvin Minsky), Psychological (Wheel of emotions by Plutchik), Neurobiological: (Cube of emotions by L\"{o}vheim)}

\cventry{2006-2014}{Software and solution Architect in Fujitsu GDC Russia}{}{}{}{Leading several research projects: affective computing, machine cognition, code generation automation, natural language processing. Software Design architect in different projects based on: Scala, OpenCog.RelEx, 
Neo4j, OpenCog.PLN, Stanford Parser, open NARS, MinorThird, Java, EJB, Hibernate, Spring, IBM MQ, Oracle BPEL}{}{}

\section{Education Background}

\cventry{2000}{PHD degree in Math modelling of Electromagnetic fields in plasma}{Kazan State Technological University,
Russia}{Faculty of enterprise processes management}{}{Thesis: \emph{Electromagnetic picture of high frequency plasma}}{}

\section{Teaching Experience}


\cvitem{2016-Now, Kazan Federal University}{\small Affective computation: the extended view on the emotions reimplementation problem in a computational system including: philosophical, psychological, neurobiological and computational perspectives. The course starts from birds eye view on the problem, carries on to neurobiological details of a neuromodulation and psychological models of emotions, then progresses into philosophical questions of consciousness and thinking and ends up with cognitive architectures and realistic neural networks review.}

\cvitem{2014-Now, Kazan Federal University}{\small Software Design Architecture: starting from basics of software design and UML to principles and design patterns}

\cvitem{2014-2015, Innopolis University}{\small Affective computation: the extended view on the emotions reimplementation problem including: philosophical, psychological, neurobiological perspectives}

\section{Projects}

\cvitem{NEUCOGAR (2014-now)}{\small Neurobiologically inspired cognitive architecture for simulation of neurobiologically plausible emotions in a computational system}
\cvitem{CellCircuit (2015-now)}{\small Nanometers neural activity sensors to be placed, powered and radiate radio frequency in neuronal cells}
\cvitem{QME (2015-now)}{\small Quantitative model of emotions based on neuromodulatory model of affects implemented in the realistic neuronal network NEST}
\cvitem{BioDynamo (2015-now)}{\small Highly distributed simulation framework: realistic neural network of a mammalian growing brain}
\cvitem{Robot dream (2015-now)}{Two phase emotional experience collection and behavioral strategy update based 
on realistic neural network}
\cvitem{TU (2012-now)}{\small Thinking-Understanding. The approach and implementation of the intelligent system for a help desk automation based on: natural language processing (NLP), probabilistic reasoning and ``Model of six'' by Marvin Minsky}
\cvitem{Menta (2011-2012)}{\small The framework for automatic software application construction via genetic algorithms and NLP}
\cvitem{IDP (2010-2011)}{\small Intellectual document processing, the project for data-mining of unstructured documents via NLP} 

\section{Language Competence}

\cvcomputer{Russian}{mother tongue}{English}{Full professional proficiency}


\section{Professional Training}

\cvitem{2014}{\small Teaching Excellence 2014, Carnegie Mellon University}
\cvitem{2009}{\small Software Architecture}
\cvitem{2008}{\small Software Requirements Analysis}
\cvitem{2007}{\small Managing Software Project Team}

\section{Computing Knowledge}

\cvcomputer{Operating Systems}{Microsoft Windows, Linux, Mac OS}{Programming}{Python, Java, Scala, Prolog, Refal}
\cvcomputer{Databases}{MSSQL, Oracle DB, PostgreSQL, Neo4j, HypergraphDB}{Development tools}{PyCharm, Jetbrains Idea, Eclipse, Visual Studio} 
\cvcomputer{Machine learning, NLP}{Rapid miner, Weka, Gate, MinorThird}{Realistic NN, Reasoner}{NEST, NARS, PLN}
\cvcomputer{OWL and mind map}{Protege, FreeMind, XMind}{Electrophy}{OpenElectrophy, Spike viewer, StimFit} 

\section{Journal papers}

\cvitem{}{\small Vallverd\"{u} J., Talanov M., Distefano S., Mazzara M., Manca M., Tchitchigin, A. - \emph{NEUCOGAR: A Neuromodulating Cognitive Architecture for Biomimetic Emotional AI}, International Journal of Artificial Intelligence (IJAI) 2016, 01/2016}


\cvitem{}{\small Vallverd\"{u} J., Talanov M., Distefano S., Mazzara M., Tchitchigin A., Nurgaliev I. - \emph{A cognitive architecture for the implementation of emotions in computing systems}, Biologically Inspired Cognitive Architectures, (BICA) 2016, 15/2016}

\cvitem{}{\small Toshchev A., Talanov M. - \emph{Coping and High Level Emotions Aspects of Computational Emotional Thinking}, International Journal of Synthetic Emotions (IJSE) 2015, 06/2015}

\cvitem{}{\small Toshchev A., Talanov M. - \emph{Computational emotional thinking and virtual neurotransmitters}, International Journal of Synthetic Emotions (IJSE), 2014, 05/2014}

\section{Conference and Workshop Papers}

 \cvitem{}{\small Bridges M. W., Distefano S., Mazzara M., Minlebaev M., Talanov M., Vallverd\"{u} J. - \emph{Towards Anthropo-Inspired Computational Systems: The $P^3$ Model}, Agent and Multi- Agent Systems: Technologies and Applications: 9th KES International Conference, KES-AMSTA 2015 Sorrento, Italy, June 2015, Proceedings (Smart Innovation, Systems and Technologies), Sorento, Italy, 2015}

 \cvitem{}{\small Toshchev A., Talanov M. - \emph{Thinking Lifecycle as an Implementation of Machine Understanding in Software Maintenance Automation Domain}, Agent and Multi-Agent Systems: Technologies and Applications: 9th KES International Conference, KES-AMSTA 2015 Sorrento, Italy, June 2015, Proceedings (Smart Innovation, Systems and Technologies), Sorento, Italy, 2015}

\cvitem{}{\small Talanov M., Vallverd\"{u} J., Distefano S., Mazzara M., Delhibabu,R. - \emph{Neuromodulating Cognitive Architecture: Towards Biomimetic Emotional AI}, IEEE 29th International Conference on Advanced Information Networking and Applications 2015, Gwangju, Korea, March 2015}

\cvitem{}{\small Toshchev A. Talanov M. - \emph{Architecture and realization of intellectual agent for automatic incident processing using the artificial intelligence and semantic networks}, Electronic libraries, Kazan, Russia, 2014}

\cvitem{}{\small Toshchev A. Talanov M. Krehov A. Khasianov A. - \emph{Thinking-Understanding approach in IT maintenance domain automation}, Global Journal on Technology (WCIT-2012), Barcelona, Spain, 2012}

\section{Conference and Workshop Activities}

\cvitem{2013}{\small AINL}
\cvitem{2015}{\small AMSTA}
\cvitem{2015}{\small AINA}
\cvitem{2015}{\small BICA}
\cvitem{2016}{\small Fierces on BICA}
\cvitem{2016}{\small AOC@AMSTA}

\end{document}
