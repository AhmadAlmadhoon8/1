%% start of file `jdoe_classic.tex'.
%% Copyright 2006 Xavier Danaux.
%
% This work may be distributed and/or modified under the
% conditions of the LaTeX Project Public License version 1.3c,
% available at http://www.latex-project.org/lppl/.

\documentclass{moderncv}
\usepackage{url}

% moderncv styles
%\moderncvstyle{casual}       % optional argument are 'nocolor' (black & white cv) and 'roman' (for roman fonts, instead of sans serif fonts)
\moderncvstyle{classic}       % idem

% character encoding
\usepackage[utf8]{inputenc}   % replace by the encoding you are using
%
% personal data (the given example is exhaustive; just give what you want)
\firstname{} \familyname{}
%\title{Highly-creative cognitive researcher}
%\address{Via Ciro Menotti, 15 – 47042, Cesenatico (FC), Italy}  % for classic style
%%\address{Via Ciro Menotti, 15 – 47042, Cesenatico (FC), Italy} % for casual style
\address{Max Talanov}
\phone{+7 962 571 8296} 
\email{max.talanov@gmail.com}
\extrainfo{\url{https://www.researchgate.net/profile/Max_Talanov}}
\extrainfo{\url{www.linkedin.com/in/max-talanov-a004aa16}}
\photo[100pt]{Talanov_Max_2012} % also optional, and the optional argument is the height the picture must be resized to

%\quote{Any intelligent fool can make things bigger, more complex, and more violent. It takes a touch of genius -- and a lot of courage -- to move in the opposite direction.}% also optional
%\quote{Video meliora proboque, deteriora sequor (Ovidio, Metamorphosis, VII, 20)}
%\renewcommand{\listsymbol}{{\fontencoding{U}\fontfamily{ding}\selectfont\tiny\symbol{'102}}} % define another symbol to be used in front of the list items

% the ConTeXt symbol
%\def\ConTeXt{%
 % C%
  %\kern-.0333emo%
  %\kern-.0333emn%
  %\kern-.0667em\TeX%
  %\kern-.0333emt}

% slanted small caps (only with roman family; the sans serif font doesn't exists :-()
%\usepackage{slantsc}
%\DeclareFontFamily{T1}{myfont}{}
%\DeclareFontShape{T1}{myfont}{m}{scsl}{ <-> cork-lmssqbo8}{}
%\usefont{T1}{myfont}{m}{scsl}Testing the font

% command and color used in this document, independently from moderncv
%\definecolor{see}{rgb}{0.5,0.5,0.5}% for web links
%\newcommand{\up}[1]{\ensuremath{^\textrm{\scriptsize#1}}}% for text subscripts

%----------------------------------------------------------------------------------
%            content
%----------------------------------------------------------------------------------
\begin{document}
%\maketitle
%\makequote
\makecvtitle

%\section{Personal Information}
%\cvitem{Name}{\small Max Talanov}
%\cvitem{Passport}{\small Russia}
%\cvitem{Date of birth}{\small 12 April 1974}
%\cvitem{Contacts}{\small mtalanov@it.kfu.ru (preferred), max.talanov@gmail.com}{}
%\cvitem{Research Gate}{https://www.researchgate.net/profile/Max\char`_Talanov}
 
\section{Personal Profile}

\cventry
{Bio}{Highly-creative cognitive researcher}{}{}{}{Has experience in: affective computing, computational neurobiology, brain simulations, machine cognition, natural language processing, probabilistic reasoning. Currently he has the position of the head of the Intellectual Robotics department of the Information Technology and Information Systems institute (ITIS) of Kazan Federal University, where he runs cross-disciplinary projects in: simulation of emotions, human-robot interface, bio-electronics, brain simulation framework, machine cognition and natural language processing. Has industrial experience as software architect and team leader for 16 years in international projects in Fujitsu.}

\cvitem
{Strengths}{Highly creative, ideas generator, good at managing team research in multy-disciplinary research, has cross-disciplinary understanding in: cognitive science, psychology, neuroscience, philosophy, computer science, IT, electronics}
%{Weaknesses}{Unable to work for a long time on a trivial tasks, has difficulties to work with demotivated team}

\section{Computing Professional Experience}

\cventry{current}{Head of the Intellectual Robotics department in Kazan Federal University}{}{Kazan, Russian Federation}{}{Management of research activities: planning, funding, publications strategy, publicity and international networking management. Management of the ``Machine cognition'' research lab with several scientific projects (see below).}

\cventry{current}{Head of Machine Cognition lab}{}{Kazan, Russian Federation}{}{Creation and management of the scientific group; leading research in several directions: affective computation, neurobiologically inspired systems, computational neurobiology; grant applications, publication activities, managing master students, international communications and collaborations, joint lab activities. In collaboration with research centers: CERN OpenLab V, Lanzhou University School of information science and engineering, Samsung Research Center; scientists: Roustem Khazipov, Jordi Vallverd\"{u}, Hu Bin, Philip Moore, Salvatore Distefano, Manuel Mazzara, Marat Minlebaev, Pei Wang.}

\cventry{2014-2015}{Lecturer at Innopolis University}{}{Kazan, Russian Federation}{}{Lecturing Affective computation course from three perspectives: Philosophical (Model of six by Marvin Minsky), Psychological (Wheel of emotions by Plutchik), Neurobiological: (``Cube of emotions'' by L\"{o}vheim).}

\cventry{2006-2014}{Software and solution Architect in Fujitsu GDC Russia}{}{}{}{Leading several research projects: affective computing, machine cognition, code generation automation, natural language processing. Software Design architect in different projects based on: Scala, OpenCog.RelEx, 
Neo4j, OpenCog.PLN, Stanford Parser, open NARS, MinorThird, Java, EJB, Hibernate, Spring, IBM MQ, Oracle BPEL}{}{}

\section{Teaching Experience}

\cvitem{2016-Now, Kazan Federal University}{\small Affective computation: the extended view on the emotions and reimplementation in a computational system problem  including: philosophical, psychological, neurobiological and computational perspectives. The course starts from birds eye view on the problem, carries on to neurobiological details of a neuromodulation and psychological models of emotions, then progresses into philosophical questions of consciousness and thinking and ends up with cognitive architectures and realistic neural networks review.}

\cvitem{2014-Now, Kazan Federal University}{\small Software Design Architecture: the course for bachelor students, intended to be starting point from basics of software design and UML to principles and design patterns, with extended use of practical examples. During the course students should develop project using principles of design studied.}

\cvitem{2014-2015, Innopolis University}{\small Affective computation: the extended view on the emotions and reimplementation in a computational system problem  including: philosophical, psychological, neurobiological and computational perspectives. The course starts from birds eye view on the problem, carries on to neurobiological details of a neuromodulation and psychological models of emotions, then progresses into philosophical questions of consciousness and thinking and ends up with cognitive architectures and realistic neural networks review.}

\section{Projects}
\cvitem{BioDynaMo (2015-now)}{\small Highly distributed simulation framework of the biologically plausible dynamic, growing neural network of a mammalian growing brain, joint with CERN, New castle university, Intel, Innopolis University}
\cvitem{NEUCOGAR (2014-now)}{\small Neurobiologically inspired cognitive architecture for simulation of neurobiologically plausible emotions in a computational system}
\cvitem{QME (2015-now)}{\small Quantitative model of emotions based on neuromodulatory model of affects implemented in the realistic neuronal network NEST}
\cvitem{Robot dream (2015-now)}{Two phase emotional experience collection and behavioral strategy update based 
on realistic neural network}
\cvitem{GDP (2015-now)}{Simulation of hippocampus Giant Depolarizing Potentials in nerobiologically plausible computational models}
\cvitem{TU (2012-now)}{\small Thinking-Understanding. The approach and implementation of the intelligent system for a help desk automation based on: machine cognition: natural language processing (NLP), probabilistic reasoning and ``Model of six'' by Marvin Minsky}
%\cvitem{CellCircuit (2015-now)}{\small Nanometers neural activity sensors to be placed, powered and radiate radio frequency in neuronal cells}
\cvitem{Menta (2011-2012)}{\small The framework for automatic software application construction via genetic algorithms and NLP}
\cvitem{IDP (2010-2011)}{\small Intellectual document processing, the project for data-mining of unstructured documents via NLP} 

\section{Prizes and Awards}

2005 - Received honorary diploma of the Sun microsystems, Java projects competition, for the MILK - domain specific language for web sites creation.

\section{Grants}

My projects received funding according to the Russian Government Program of Competitive Growth of Kazan Federal University.

\section{Granted patent(s)}

Thinking-Understanding the registration of intellectual property of software product or technology.

\section{Papers}

\cvitem{}{\small Vallverd\"{u} J., Talanov M., Distefano S., Mazzara M., Manca M., Tchitchigin A. - \emph{NEUCOGAR: A Neuromodulating Cognitive Architecture for Biomimetic Emotional AI}, International Journal of Artificial Intelligence (IJAI) 2016, 01/2016}


\cvitem{}{\small Vallverd\"{u} J., Talanov M., Distefano S., Mazzara M., Tchitchigin A., Nurgaliev I. - \emph{A cognitive architecture for the implementation of emotions in computing systems}, Biologically Inspired Cognitive Architectures, (BICA) 2016, 15/2016}

\cvitem{}{\small Toshchev A., Talanov M. - \emph{Coping and High Level Emotions Aspects of Computational Emotional Thinking}, International Journal of Synthetic Emotions (IJSE) 2015, 06/2015}

\cvitem{}{\small Toshchev A., Talanov M. - \emph{Computational emotional thinking and virtual neurotransmitters}, International Journal of Synthetic Emotions (IJSE), 2014, 05/2014}

%\section{Conference and Workshop Papers}

\cvitem{}{\small Wang P., Talanov M., Hammer P. - \emph{The Emotional Mechanisms in NARS}, In press, New York, USA, 2016}

\cvitem{}{\small Toshchev A., Talanov M., Distefano S. - \emph{Evolution of thinking models in automatic incident processing systems}, Agent and Multi-Agent Systems: Technologies and Applications: 10th KES International Conference, KES-AMSTA 2016 Tenerife, June 2016, Proceedings (Smart Innovation, Systems and Technologies), Tenerife, Spain, 2016}

\cvitem{}{\small Tchitchigin A., Safina L., Talanov M., Mazzara M. - \emph{Robot Dream}, Agent and Multi-Agent Systems: Technologies and Applications: 10th KES International Conference, KES-AMSTA 2016 Tenerife, June 2016, Proceedings (Smart Innovation, Systems and Technologies), Tenerife, Spain, 2016}


\cvitem{}{\small Leukhin, A.,  Talanov, M.,  Sozutov, I.,  Vallverd\"{u} J,  Toschev, A. - \emph{Simulation of a fear-like state on a model of dopamine system of rat brain}, Advances in Intelligent Systems and Computing Volume 449, 2016, Pages 121-126, 1st International Early Research Career Enhancement School on Biologically Inspired Cognitive Architectures, FIERCES on BICA 2016; New York; United States, 2016}

\cvitem{}{\small Max Talanov, Alexander Toschev, - Appraisal, Coping and High Level Emotions Aspects of Computational Emotional Thinking,  International Journal of Synthetic Emotions (IJSE), 2015}

\cvitem{}{\small Bridges M. W., Distefano S., Mazzara M., Minlebaev M., Talanov M., Vallverd\"{u} J. - \emph{Towards Anthropo-Inspired Computational Systems: The $P^3$ Model}, Agent and Multi- Agent Systems: Technologies and Applications: 9th KES International Conference, KES-AMSTA 2015 Sorrento, Italy, June 2015, Proceedings (Smart Innovation, Systems and Technologies), Sorento, Italy, 2015}

\cvitem{}{\small Toshchev A., Talanov M. - \emph{Thinking Lifecycle as an Implementation of Machine Understanding in Software Maintenance Automation Domain}, Agent and Multi-Agent Systems: Technologies and Applications: 9th KES International Conference, KES-AMSTA 2015 Sorrento, Italy, June 2015, Proceedings (Smart Innovation, Systems and Technologies), Sorento, Italy, 2015}

\cvitem{}{\small Talanov M., Vallverd\"{u} J., Distefano S., Mazzara M., Delhibabu,R. - \emph{Neuromodulating Cognitive Architecture: Towards Biomimetic Emotional AI}, IEEE 29th International Conference on Advanced Information Networking and Applications 2015, Gwangju, Korea, March 2015}

\cvitem{}{\small Toshchev A. Talanov M. - \emph{Architecture and realization of intellectual agent for automatic incident processing using the artificial intelligence and semantic networks}, Electronic libraries, Kazan, Russia, 2014}

\cvitem{}{\small Toshchev A. Talanov M. Krehov A. Khasianov A. - \emph{Thinking-Understanding approach in IT maintenance domain automation}, Global Journal on Technology (WCIT-2012), Barcelona, Spain, 2012}

\cvitem{}{\small Talanov M. Krehov A. Makhmutov A. - \emph{Automating programming via concept mining, probabilistic reasoning over semantic knowledge base of SE domain}, 2010 6th Central and Eastern European Software Engineering Conference, CEE-SECR 2010, Article number 5783147, Pages 30-35, 2010 6th Central and Eastern European Software Engineering Conference, CEE-SECR 2010; Moscow; Russian Federation; 13 October 2010 through 15 October 2010}

\section{Conference and Workshop Activities}

\cvitem{2010}{\small CEE-SECR}
\cvitem{2013}{\small AINL}
\cvitem{2015}{\small AOC@AMSTA}
\cvitem{2015}{\small AINA}
\cvitem{2015}{\small BICA}
\cvitem{2016}{\small Fierces on BICA}
\cvitem{2016}{\small AOC@AMSTA}
\cvitem{2016}{\small AGI}
\cvitem{2016}{\small BICA}

Took part in organisation of special sessions of AOC@AMSTA-2016 and AOC@AMSTA-2016.
Took part in organisation of local series of Software engineering, seminars AKSES-2014.

\section{Professional Training}

\cvitem{2014}{\small Teaching Excellence, Carnegie Mellon University}
\cvitem{2009}{\small Software Architecture}
\cvitem{2008}{\small Software Requirements Analysis}
\cvitem{2007}{\small Managing Software Project Team}

\section{Computing Knowledge}

\cvitem{OS}{Microsoft Windows, Linux, Mac OS}
\cvitem{Programming}{Scala, Python, Java, Prolog, Refal, C++, Haskell, PLSQL, TSQL, Java SE, EJB, Hibernate, Spring, Seam, IBM MQ, Oracle BPEL, JBoss, Tomcat, Web Services}
\cvitem{Databases}{MSSQL, Oracle DB, PostgreSQL, Neo4j, HypergraphDB}
\cvitem{Development}{Jetbrains PyCharm, Jetbrains Idea, Eclipse, Visual Studio, Maven, Emacs, ArgoUML, Visual Paradigm, Rational Rose, Enterprise Architect, QTCreator, Oracle JDeveloper} 
\cvitem{ML, NLP}{Rapid miner, Weka, OpenCog.RelEx, Stanford parser, Gate, MinorThird}
\cvitem{Realistic NN, Reasoner}{NEST, open NARS, OpenCog.PLN, Pellet}
\cvitem{OWL and mind map}{Protege, FreeMind, XMind}
\cvitem{Electrophy}{OpenElectrophy, Spike viewer, StimFit} 

\section{Education Background}

\cventry{2000}{PHD degree in Math modelling of Electromagnetic fields in plasma}{Kazan State Technological University,
Russia}{Faculty of enterprise processes management}{}{Thesis: \emph{Electromagnetic picture of high frequency plasma}}{}

\section{Language Competence}

\cvcomputer{English}{Full professional proficiency}{Russian}{mother tongue}

\end{document}
